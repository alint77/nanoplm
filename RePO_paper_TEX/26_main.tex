%%%%%%%% ICML 2026 EXAMPLE LATEX SUBMISSION FILE %%%%%%%%%%%%%%%%%

\documentclass{article}

% Recommended, but optional, packages for figures and better typesetting:
\usepackage{microtype}
\usepackage{graphicx}
\usepackage{subcaption}
\usepackage{booktabs} % for professional tables

% hyperref makes hyperlinks in the resulting PDF.
% If your build breaks (sometimes temporarily if a hyperlink spans a page)
% please comment out the following usepackage line and replace
% \usepackage{icml2026} with \usepackage[nohyperref]{icml2026} above.
\usepackage{hyperref}


% Attempt to make hyperref and algorithmic work together better:
\newcommand{\theHalgorithm}{\arabic{algorithm}}

% Use the following line for the initial blind version submitted for review:
\usepackage[preprint]{icml2026}

% For preprint, use
% \usepackage[preprint]{icml2026}

% If accepted, instead use the following line for the camera-ready submission:
% \usepackage[accepted]{icml2026}

\usepackage{amsmath}
\usepackage{amssymb}
\usepackage{mathtools}
\usepackage{amsthm}


% if you use cleveref..
\usepackage[capitalize,noabbrev]{cleveref}

%%%%%%%%%%%%%%%%%%%%%%%%%%%%%%%%
% Custom Package
%%%%%%%%%%%%%%%%%%%%%%%%%%%%%%%%

\usepackage{float}
% \usepackage[table]{xcolor}
\usepackage[export]{adjustbox}
% \usetikzlibrary{shadows.blur}
\usepackage[most]{tcolorbox}
\usepackage{array}
\usepackage{soul}
\usepackage[super]{nth}
\usepackage{mdframed}
%\usepackage{ulem}
\usepackage{inconsolata}
\usepackage{bm}
\usepackage{mathrsfs}
\usepackage{epsfig}
\usepackage{graphicx}
\usepackage{booktabs}
\usepackage{verbatim}
\usepackage{tipa}
\usepackage{url}
\usepackage{amsfonts}       % blackboard math symbols
\usepackage{nicefrac}       % compact symbols for 1/2, etc.
\usepackage{xcolor}         % colors
\usepackage{tcolorbox}
\usepackage{algorithm}
\usepackage{listings}
\usepackage{inconsolata}
\usepackage{bm}
\usepackage{amsthm,amsmath,amssymb}
\usepackage{mathrsfs}
\usepackage{epsfig}
\usepackage{booktabs}
\usepackage{verbatim}
\usepackage{enumitem,kantlipsum}
\usepackage{lipsum}
\usepackage{tipa}
% \usepackage{algorithmicx}
% \usepackage{algpseudocode}
\usepackage{multirow,makecell}
\usepackage{arydshln}
% \usepackage{xcolor}
\usepackage{colortbl}

\usepackage{color} 
\usepackage{dsfont}
\usepackage{tikz}
\usepackage{wrapfig}
\usepackage{xspace}
\usepackage{ifthen}
% \usepackage{subcaption}
\usepackage{pdflscape}
\usepackage{pifont}
\usepackage{bbm}
\usepackage{dsfont}
\usepackage{graphicx}
\usepackage{subcaption}  % <-- Make sure this is used, not 'subfigure'
\usepackage{caption}
\newcommand{\xmark}{\ding{55}}
\newcommand{\cmark}{\ding{51}}
\newtcolorbox{AIbox}[2][]{aibox,title=#2,#1}
\usepackage{csquotes}

\usepackage{hyperref}
\tcbset{
  aibox/.style={
    width=450pt,
    top=10pt,
    % colback=white,
    colframe=black,
    colbacktitle=black,
    enhanced,
    center,
    attach boxed title to top left={yshift=-0.1in,xshift=0.15in},
    boxed title style={boxrule=0pt,colframe=white,},
  }
}
\definecolor{forestgreen}{rgb}{0.13, 0.55, 0.13}

\newcommand{\seq}[1]{\boldsymbol{#1}}
\usepackage{color} 
\definecolor{mypink2}{RGB}{219, 48, 122}
\definecolor{orange}{RGB}{255, 147, 00}
\definecolor{jrcolor}{RGB}{100, 150, 225}
\definecolor{jrcomment}{RGB}{70, 200, 150}
\definecolor{grey}{RGB}{166, 166, 166}

\newcommand{\PredSty}[1]{\textnormal{\ttfamily\color{mygreen!90!black}#1}\unskip}
\newcommand{\Xmat}[0]{{{\bf X}}}
\definecolor{mygreen}{HTML}{3cb44b}

\newcommand{\ind}[1]{\mathds{1}{\{#1\}}} % define the indicator function
\DeclareMathOperator*{\argmax}{arg\,max}
\DeclareMathOperator*{\argmin}{arg\,min}

\definecolor{mypink2}{RGB}{219, 48, 122}
\definecolor{orange}{RGB}{255, 147, 00}
\definecolor{jrcolor}{RGB}{100, 150, 225}
\definecolor{jrcomment}{RGB}{70, 200, 150}
\newcommand{\hyli}[1]{\textcolor{mypink2}{\textbf{[By hyli: #1]}}}
\newcommand{\yahui}[1]{\textcolor{jrcolor}{\textbf{[By yahui: #1]}}}
\newcommand{\lycui}[1]{\textcolor{jrcomment}{\textbf{[By jnx: #1]}}}
\newcommand{\cd}[1]{\textcolor{orange}{\textbf{[By Taro: #1]}}}

% define commands for shapes
\newcommand{\myoval}{\protect\tikz \protect\draw (0,0) ellipse (0.2cm and 0.1cm);}
\newcommand{\mycircle}{\protect\tikz \protect\draw (0,0) circle (0.15cm);}
\newcommand{\myrectangle}{\protect\tikz \protect\draw (0,0) rectangle (0.4cm,0.2cm);}
\newcommand{\mydiamond}{\protect\tikz \protect\draw (0,0) -- (0.2cm,0.1cm) -- (0.4cm,0cm) -- (0.2cm,-0.1cm) -- cycle;}


\NewDocumentCommand{\myparagraph}{m}{%
  \par\vspace{1.5ex}%
  \noindent\textbf{#1}\hspace{0.5em}%
}

\usepackage{cleveref}
\crefname{section}{§}{§§}
\Crefname{section}{§}{§§}

\definecolor{NavyBlue}{rgb}{0.1, 0.4, 0.8}
\hypersetup{colorlinks = true, linkcolor = NavyBlue,
            urlcolor  = gray,
            citecolor = NavyBlue,
            anchorcolor = NavyBlue}

\newcommand{\implname}{\textsc{RePo}\xspace}
\usepackage[dvipsnames]{xcolor}

%%%%%%%%%%%%%%%%%%%%%%%%%%%%%%%%
% THEOREMS
%%%%%%%%%%%%%%%%%%%%%%%%%%%%%%%%
\theoremstyle{plain}
\newtheorem{theorem}{Theorem}[section]
\newtheorem{proposition}[theorem]{Proposition}
\newtheorem{lemma}[theorem]{Lemma}
\newtheorem{corollary}[theorem]{Corollary}
\theoremstyle{definition}
\newtheorem{definition}[theorem]{Definition}
\newtheorem{assumption}[theorem]{Assumption}
\theoremstyle{remark}
\newtheorem{remark}[theorem]{Remark}

% Todonotes is useful during development; simply uncomment the next line
%    and comment out the line below the next line to turn off comments
%\usepackage[disable,textsize=tiny]{todonotes}
\usepackage[textsize=tiny]{todonotes}

% The \icmltitle you define below is probably too long as a header.
% Therefore, a short form for the running title is supplied here:
% \icmltitlerunning{Submission and Formatting Instructions for ICML 2026}

\begin{document}

\twocolumn[
\icmltitle{\implname: Language Models with Context Re-Positioning}

  % It is OKAY to include author information, even for blind submissions: the
  % style file will automatically remove it for you unless you've provided
  % the [accepted] option to the icml2026 package.

  % List of affiliations: The first argument should be a (short) identifier you
  % will use later to specify author affiliations Academic affiliations
  % should list Department, University, City, Region, Country Industry
  % affiliations should list Company, City, Region, Country

  % You can specify symbols, otherwise they are numbered in order. Ideally, you
  % should not use this facility. Affiliations will be numbered in order of
  % appearance and this is the preferred way.
  \icmlsetsymbol{equal}{*}

  \begin{icmlauthorlist}
    \icmlauthor{Huayang Li}{sakana,naist}
    \icmlauthor{Tianyu Zhao}{sakana}
    \icmlauthor{Richard Sproat}{sakana}
  \end{icmlauthorlist}

  \icmlaffiliation{sakana}{Sakana AI, Japan}
  \icmlaffiliation{naist}{Nara Institute of Science and Technology (NAIST), Japan}

  \icmlcorrespondingauthor{Huayang Li}{li.huayang.lh6@is.naist.jp}
  \icmlcorrespondingauthor{Tianyu Zhao}{tianyu@sakana.ai}
  \icmlcorrespondingauthor{Richard Sproat}{rws@sakana.ai}
  % You may provide any keywords that you find helpful for describing your
  % paper; these are used to populate the "keywords" metadata in the PDF but
  % will not be shown in the document
  \icmlkeywords{Machine Learning, ICML}

  \vskip 0.3in
]

% this must go after the closing bracket ] following \twocolumn[ ...

% This command actually creates the footnote in the first column listing the
% affiliations and the copyright notice. The command takes one argument, which
% is text to display at the start of the footnote. The \icmlEqualContribution
% command is standard text for equal contribution. Remove it (just {}) if you
% do not need this facility.

% Use ONE of the following lines. DO NOT remove the command.
% If you have no special notice, KEEP empty braces:
\printAffiliationsAndNotice{}  % no special notice (required even if empty)
% Or, if applicable, use the standard equal contribution text:
% \printAffiliationsAndNotice{\icmlEqualContribution}

\begin{abstract}
In-context learning is fundamental to modern Large Language Models (LLMs); however, prevailing architectures impose a rigid and fixed contextual structure by assigning linear or constant positional indices.
Drawing on Cognitive Load Theory (CLT), we argue that this uninformative structure increases extraneous cognitive load, consuming finite working memory capacity that should be allocated to deep reasoning and attention allocation.
To address this, we propose \implname, a novel mechanism that reduces extraneous load via context re-positioning. Unlike standard approaches, \implname utilizes a differentiable module, $f_\phi$, to assign token positions that capture contextual dependencies, rather than replying on pre-defined integer range. By continually pre-training on the OLMo-2 1B backbone, we demonstrate that \implname significantly enhances performance on tasks involving noisy contexts, structured data, and longer context length, while maintaining competitive performance on general short-context tasks. Detailed analysis reveals that \implname successfully allocate higher attention to distant but relevant information, assign positions in dense and non-linear space, and capture the intrinsic structure of the input context. 
Our code is available at \url{https://github.com/SakanaAI/repo}.
\end{abstract}
\section{Introduction}


\begin{figure}[]
    \begin{center}
        \includegraphics[width=0.95\columnwidth]{figs/repo_overall.png}
    \end{center}
    \caption{Overall performance on four evaluation dimensions.\label{fig:overall}}
\end{figure}

The emergence of large language models (LLMs) \cite{brown2020language} has enabled a wide range of applications, including few-shot learning \cite{brown2020language}, retrieval-augmented generation \cite{lewis2020retrieval, yao2023react}, and agentic systems \cite{schick2023toolformer, park2023generative}. At the core of these applications is in-context learning \cite{shen2024position}, a form analogous to human \emph{working memory}, where information within a limited context window is temporarily stored and processed to solve a task. Consequently, exploring how to effectively utilize context information has become a fundamental research line in the LLM era \cite{wei2022chain, weston2023system, chen2023extending}.

Recent studies show that an LLM’s ability to leverage contextual information is strongly influenced by its position encoding scheme \cite{vaswani2017attention, press2021train, su2024roformer}. Most LLMs impose a fixed contextual structure by assigning tokens consecutive integer indices from $0$ to $L-1$ \cite{vaswani2017attention} or a constant index $a$ for all tokens \cite{kazemnejad2023impact}. These indices are then integrated into a model through position encoding functions, enforcing a rigid organization of context.


Although fixed position assignments have become the \emph{de facto} standard, they deviate from how human working memory processes information. According to Cognitive Load Theory (CLT), the capacity of working memory during problem solving can be consumed by costs arising from how information is organized and presented, referred to as \textit{extraneous load} \cite{sweller1994cognitive, paas2003cognitive}. CLT studies  suggest that humans can actively reduce this extraneous load by reorganizing context, e.g., grouping related information into meaningful chunks \cite{miller1956magical} or removing irrelevant details from instructions \cite{sweller1994cognitive}. Since working memory capacity is fixed, the capacity saved through such reorganization can be reallocated to deeper reasoning processes associated with the \textit{the germane load}, thereby improving problem-solving performance \cite{sweller1994cognitive}.

However, the critical ability to reorganize and restructure contextual information \cite{vaswani2017attention, yang2025qwen3, dubey2024llama} is absent from the architectural design of modern LLMs. From the perspective of CLT \cite{sweller1994cognitive, paas2003cognitive}, rigid linear or constant position structures can be interpreted as introducing additional extraneous load, which in turn impairs attention allocation and deeper contextual reasoning (i.e., germane processing). As a consequence, tasks that require strong long-range or fine-grained contextual dependencies, e.g., needle-in-a-haystack (NIAH) problems \cite{kamradt2023needle} or question answering under highly diluted contexts \cite{hsiehruler}, exhibit notable performance degradation, mirroring the effects predicted by CLT under high extraneous load. Moreover, from a probabilistic standpoint, these position assignment strategies, essentially uniform distributions over fixed integer ranges,  are
the least informative organizations of context and therefore limit representational efficiency.   


We propose an internal mechanism for LLMs to reduce \textit{extraneous load} by re-organizing the positions of tokens.
% , thereby conserving the finite working memory capacity for beneficial \textit{germane load}. 
We formalize this process, termed \emph{context Re-Positioning} (\implname), as learning non-constrained position values based on information relevance of tokens, instead of using the fixed linear positions in prior work. To this end, we introduce a differentiable module $f_\phi$, which assigns a position value in continuous space for each token based on its hidden state. The $f_\phi$ can be independently learned for each attention head of an LLM. Trained on general data, $f_\phi$ learns to re-position tokens free from conventional constraints like monotonicity or integer values. The continuity of modern position encoding functions, e.g., RoPE \cite{su2024roformer} and ALiBi \cite{press2021train} methods, is key to the end-to-end optimization of $f_\phi$, as it allows these assigned positions to be integrated in a differentiable manner. 


We find that LLMs using the \implname method achieve consistent performance gains on tasks involving noisy context, structured data, and longer context. In our experiments, we continually pre-trained LLMs with the \implname method and several baselines based on the OLMo-2 1B model.
% \cite{walsh2} as the backbone for 50B tokens. 
Within the training context length, our \implname method outperforms other baselines by at least 6.24 and 1.16 points on noisy context and structured data tasks, respectively. In addition, when extending the testing context length to 16K tokens using the YaRN \cite{peng2024yarn} method, our \implname method outperforms other baselines by at least 13.25 EM points on the QA and Needle-in-the-haystack (NIAH) tasks and 5.48 points on LongBench \cite{bai-etal-2024-longbench}. Alongside these performance gains, \implname achieves comparable results on extensive general benchmarks \cite{wang2024mmlu, clark2018think, zellers-etal-2019-hellaswag}, which are typically short and require little reorganization.

To better understand whether our method aligns with the CLT-based motivation and where the performance gains of the \implname method come from, we conducted a series of detailed analyses. \textbf{First}, to evaluate how different methods handle long-range dependencies, we compared the attention distributions across methods, particularly focusing on their treatment of distant but relevant information. In the NIAH task (Section~\ref{ssec:niah}), we observed that \implname allocates more attention to the most critical ``needle'' tokens compared to the baseline methods, while directing less attention to the nearest ``query'' tokens. This behavior breaks the typical locality bias \cite{yang2025rope, press2021train}, dynamically adjusting based on the context. \textbf{Second}, the positions assigned by \implname exist in a denser, more non-linear space, which is critical for enhancing its generalization to longer contexts, such as extending from 4K to 16K tokens. \textbf{Finally}, an interesting finding is that \implname learns position patterns that resemble a hybrid of previous position assignment strategies, such as assigning constant positions $a \pm \epsilon$ \cite{kazemnejad2023impact} or enforcing a monotonic position sequence \cite{vaswani2017attention, yang2025rope, press2021train}, within a given context span (Section~\ref{ssec:pos_pattern}). In our case study (Appendix~\ref{app:case_study}), we also found that the positions assigned by \implname capture the intrinsic structure of the input context (e.g., segmentation of few-shot examples). 


\section{Background\label{sec:bg}}
The position information of tokens in a context is critical for the attention mechanism of LLMs. The position of a given token is generally mapped into embeddings \cite{su2024roformer, vaswani2017attention, li2025seqpe} or biases \citep{press2021train, raffel2020exploring} through a position encoding module before the attention function. In this section, we will introduce the notations for attention mechanism and position encoding functions.


Given an input sequence $\boldsymbol{x} = (x_1, x_2, \dots, x_L)$, where each token $x_i$ is drawn from a vocabulary $\mathcal{V}$, most large language models (LLMs) process the information in $\boldsymbol{x}$ through multiple layers of self-attention-based neural networks.
In each layer, the attention score\footnote{For simplicity, we omit the design of multi-head in the attention calculation.} between tokens $x_i$ and $x_j$ is computed as follows:
\begin{align}
\mathbf{Q}, \mathbf{K}, \mathbf{V} &= \mathbf{H}\mathbf{W}^{{q, k, v}}, \label{eq:qkv}\\
\mathbf{A}_{i,j} &= \boldsymbol{q}_i^\top \boldsymbol{k}_j,
\end{align}
where $\mathbf{W}^{{q,k,v}} \in \mathbb{R}^{d \times 3d}$ projects the hidden state $\boldsymbol{h}_i \in \mathbf{H}$ of token $x_i$ into its corresponding query, key, and value representations, and $\textbf{A} \in \mathbb{R}^{L\times L}$ represents the attention scores for all token pairs.  


We use the \emph{rotary position encoding} (\textsc{RoPE}) \cite{su2024roformer} as a specific example to illustrate the position assignment and encoding procedure commonly employed in LLMs \cite{walsh2, yang2025qwen3, qwen2025qwen25technicalreport}. First, each token \(x_i\) is assigned an integer position index \(i\). This positional information is then incorporated into the model through an encoding function \(g_\theta: \mathbb{R} \rightarrow \mathbb{R}^{d \times d}\), which is a differentiable function that generates a rotation matrix. The parameter \(\theta\) represents a pre-defined frequency vector, generally frozen during training. In \textsc{RoPE}, the function \(g_\theta\) is directly applied to the query \(\boldsymbol{q}_i\) and key \(\boldsymbol{k}_j\) as follows:
\begin{align}
\mathbf{A}_{i,j}^\text{RoPE} = \boldsymbol{q}_i^\top g_\theta(j-i)\boldsymbol{k}_j \label{eq:rope},
\end{align}
where \(i\) and \(j\) are the integer position indices for tokens \(x_i\) and \(x_j\), respectively, and \(g_\theta(j-i)\) captures the relative positional relationship between any tokens with a distance of \(j-i\). 

In many related works, strict linear position assignment (i.e., 0 to $L-1$) similar to \textsc{RoPE} has become the standard approach, with various encoding functions being explored \cite{vaswani2017attention, press2021train, yang2025rope, li2025seqpe}. One notable exception is the \textsc{NoPE} method \cite{kazemnejad2023impact, yang2025rope}, which omits both position assignment and position encoding. However, we demonstrate that this approach is equivalent to applying position encoding at a constant position \(a\) for all tokens. Further discussion about the inter-connection between constant (e.g., \textsc{NoPE}) and linear position assignment (e.g., \textsc{RoPE}) strategies are provided in Appendix \ref{app:comp}.




\section{Methods}

\subsection{Overview}
The current linear context organization, which assigns consecutive integer indices to tokens, overlooks the internal structure of tokens based on their relevance. For instance, this limitation leads to noticeable performance degradation in tasks involving long-distance dependencies \cite{hsiehruler}, a problem analogous to the high extraneous load issue in Cognitive Load Theory (CLT).

The main goal of this work is to reduce the unnecessary cognitive cost raised by the oversimplified organization of the context, i.e., \textit{extraneous cognitive load}, thereby conserving the finite working memory capacity for beneficial \textit{germane load}, such as the attention mechanism. To this end, we propose a context re-positioning (\implname) module $f_\phi: \mathbb{R}^{d} \rightarrow \mathbb{R}$, which is a light-weight neural network that assigns more appropriate positions of tokens, taking into account their relevance within a given context. The assigned positions by $f_\phi$ are defined in a continuous, non-linear space, and can thus be optimized jointly with LLMs when equipped with differentiable position encoding methods \cite{vaswani2017attention, su2024roformer, press2021train}. Notably, our \implname module $f_\phi$ is prior to position encoding, where the latter aims to map assigned positions into embeddings or biases.  



\subsection{Context Re-positioning}
The context re-positioning module $f_\phi$ has two components: 1) representing the position information based on hidden states of tokens; 2) assigning real-valued positions based on the extracted position representations.





\paragraph{Position Representation}
\citet{kazemnejad2023impact} shows that position information may be entangled in original hidden states, so the first component is designed to extract the position representation from hidden states of tokens explicitly. In our implementation, we use a light-weight SwiGLU \cite{shazeer2020glu} sub-layer to achieve this goal:
\begin{equation}
\boldsymbol{r}_i = \mathrm{Swish}(\boldsymbol{h}_i\mathbf{W}^{g})\odot(\boldsymbol{h}_i\mathbf{W}^{c}),\label{eq:pos_repr}
\end{equation}
where $\boldsymbol{r}_i \in \mathbb{R}^{d_p}$ and $\boldsymbol{h}_i \in \mathbb{R}^d$ are position representation\footnote{Notably, the hidden state $\boldsymbol{h}_i$ of token $x_i$ does not explicitly encode the linear position information $i$. In our preliminary experiments, when using the raw position $i$ as an additional dimension during continue-training, the LLM pre-trained on \textsc{RoPE} quickly biases to this feature, resulting in trivial position assignment.} and hidden state of token $x_i$ respectively, $\mathbf{W}^{g},\mathbf{W}^{c} \in \mathbb{R}^{d\times d_p}$ are linear transformations for gate and content mapping, and $\mathrm{Swish}(\cdot)$ is the activation function.  Since we assume that position information can be represented with a lower dimension, we set $d_p < d$ in practice. 


\paragraph{Position Assignment} The subsequent component assigns a new position value $z_i$ for token $x_i$ in each attention head. There are a variety of modeling strategies for processing $\boldsymbol{r}_i$ with full \cite{vaswani2017attention} or limited \cite{lecun1998convolutional} access to $\boldsymbol{r}_{<i}$, but we find that a linear transformation achieves comparable performance with lower latency:
\begin{equation}
    z_i = \boldsymbol{r}_i\mathbf{W}^z,\label{eq:pos_head}
\end{equation}
where $\mathbf{W}^z \in \mathbb{R}^{d_p\times1}$. 

\paragraph{\implname Module} By combining Eq. \ref{eq:pos_repr} and \ref{eq:pos_head}, the formal definition of $f_\theta$ becomes:
\begin{align}
f_\phi(\boldsymbol{h}_i) &= \big(\mathrm{Swish}(\boldsymbol{h}_i\mathbf{W}^{g})\odot(\boldsymbol{h}_i\mathbf{W}^{c})\big)\mathbf{W}^z,
\end{align}
where $\boldsymbol{h}_i$ is the hidden state of token $x_i$, as defined in Eq. \ref{eq:qkv}.
When equipped with modern position encoding methods, e.g., RoPE, the computation of attention score becomes:
\begin{align}
    \textbf{A}^{\text{RePo}}_{i,j} &= \boldsymbol{q}_i^\top g_\theta \big(f_\phi(\boldsymbol{h}_j) - f_\phi(\boldsymbol{h}_i)\big) \boldsymbol{k}_j \nonumber\\
    &= \boldsymbol{q}_i^\top g_\theta (z_j - z_i) \boldsymbol{k}_j,
    \label{eq:repo}
\end{align}
where the position encoding function $g_\theta$ is the same as that used in \textsc{RoPE} in Eq. \ref{eq:rope}. It is worth noting that our \implname is not restricted to \textsc{RoPE} and can be easily extended to all the differentiable position encoding methods \cite{vaswani2017attention, press2021train, li2025seqpe}. In practice, we apply the position representation module (Eq. \ref{eq:pos_repr}) for each layer independently, and position assignment module (Eq. \ref{eq:pos_head}) for each attention head independently. 
% The pseudo code for the full implementation of $f_\phi$ is shown in Alg 1. 


\begin{table}
\centering
\caption{Performance on noisy context. We evaluate on subsets of RULER with noisy context, i.e., Needle-in-the-haystack (NIAH), Question Answering (QA), Aggregation (AGG), and variable tracking (VT), within the training context length (i.e., 4K tokens). 
This uses recall for NAIH, AGG, and VT tasks, and EM for the QA task as the metrics. 
We use \textbf{bold} and \underline{underline} for best and second best results.\label{tab:anti_noise}}
\resizebox{1.0\columnwidth}{!}{
\begin{tabular}{lrrrrr}\toprule
Model & NIAH  & QA   & AGG & VT & AVG.  \\\midrule
\textsc{RoPE}    & 82.56 & 57.00  & \textbf{37.98}       & 1.00              & 44.64 \\\hdashline
\textsc{NoPE}    & 74.59 & 49.00  & 22.45       & 12.20             & 39.56 \\
\textsc{R2N1}   & \underline{85.00} & \underline{59.50}   & 31.10       & 0.00              & 43.90 \\
N2R1   & 80.00 & 58.00   & 32.75       & \underline{27.00}             & \underline{49.44} \\\hdashline
\implname    & \textbf{88.25} & \textbf{61.00}   & \underline{35.05}       & \textbf{38.40}            & \textbf{55.68} \\\bottomrule
\end{tabular}
}
% \vspace{-1.2em}
\end{table}


\paragraph{Training \& Efficiency} Since many position encoding methods are differentiable \cite{su2024roformer,press2021train,li2025seqpe}, we can train an LLM with \implname-based attention (Eq.~\ref{eq:repo}) using backpropagation when equipped with such encodings. To balance efficiency and effectiveness, we apply the \implname method starting from the $l$-th layer (e.g., $l=5$) while keeping standard position encoding for the lower layers. This design choice is motivated by previous findings that the lower layers of LLMs primarily capture surface-level features that depend more on local information \cite{tenney-etal-2019-bert}, such as part-of-speech tagging and syntax, and thus benefit less from reorganization. An ablation study for the hyper-parameter is in Appendix \ref{app:ablation}.

In order not to impair the efficiency of LLMs significantly, we only use the assigned position $z_i$ and $z_j$ to affect the position encoding in attention calculation in Eq. \ref{eq:repo}, leaving the auto-regressive order of $\boldsymbol{q}_i$ (or $\boldsymbol{k}_i$) and $\boldsymbol{q}_j$ (or $\boldsymbol{k}_j$) in the context unchanged. The \implname module can be applied to each attention head independently. In principle, we can sort queries and keys in each attention head according to the assigned positions $\boldsymbol{z} = \{z_1, \dots, z_n\}$. However, under the auto-regressive language modeling paradigm, this approach requires re-computation for the KV cache at each time step, resulting in tremendous overhead for auto-regressive LLMs. Therefore, the assigned positions are only used in Eq. \ref{eq:repo}. 




\section{Experiments\label{sec:exp}}
This section presents our main experiments on general language modeling. We continually pre-train an LLM with \implname on general datasets and evaluate its performance on three types of tasks that require restructuring context. In Appendix~\ref{app:pre}, we present preliminary studies on synthetic data along with visualizations to show how \implname works.



\begin{table}[]
\centering
\caption{Performance for structured data. We evaluate on NLGraph and HybridQA datasets for graph and table data, respectively. We use exact match (EM) as metric. \textbf{Bold} and \underline{underline} denote the best and the second best results, respectively. \label{tab:struct}}
\tiny
\resizebox{0.85\columnwidth}{!}{
\begin{tabular}{lccc}\toprule
     & Graph & Table & AVG.   \\\midrule
\textsc{RoPE} & 27.43                  & 24.43              & 25.93 \\\hdashline
NoPE & \textbf{29.90}                  & 23.52              & \underline{26.71} \\
\textsc{R2N1} & 27.11                  & \underline{25.11}              & 26.11 \\
\textsc{N2R1} & 25.42                  & 23.86              & 24.64 \\\hdashline
\implname & \underline{29.03}                  & \textbf{26.70}              & \textbf{27.87} \\\bottomrule
\end{tabular}
}
\end{table}


\begin{figure*}[]
    \begin{center}
        \includegraphics[width=0.85\textwidth]{figs/repo_long_context.png}
    \end{center}
    \caption{Long-context Evaluation on RULER. YaRN \cite{peng2024yarn} is used for all RoPE layers to extend the context.  We observe consistent results on the more realistic benchmark LongBench in Table \ref{tab:lc_longbench}.\label{fig:lc_ruler}}
\end{figure*}



\subsection{Setup} We use OLMo-2 \cite{walsh2} developed by Allen Institute for AI as the backbone model, whose performance is comparable to Qwen-2.5 \cite{qwen2025qwen25technicalreport}. Its data, model weights, and code are fully open-sourced. Notably, OLMo-2 is trained with the RoPE method \cite{su2024roformer}. For all methods in our experiments, we start from the checkpoint of the OLMo-2 1B model that has completed stage-1 pre-training, and we continually pre-train it on the $50$B-token stage-2 data\footnote{\url{https://github.com/allenai/OLMo}}. The training context length is 4096 tokens. We keep the training configuration and codebase identical to those released by \citet{walsh2}.

For our \implname method, we apply it starting from the 5th layer of the OLMo-2 1B model. In each layer that uses \implname, we share the parameters for position representation transformation in Eq.~\ref{eq:pos_repr}, while learning the position assignment for each attention head independently as in Eq.~\ref{eq:pos_head}. The hidden size for the learned position representation is 256, i.e., $1/8$ of the model’s hidden size. 
As shown in \S\ref{app:efficiency}, \implname is lightweight and introduces negligible overhead to the original LLM. 
We compare \implname with baselines:
\begin{itemize}%[leftmargin=*, nosep]
    \item \textsc{RoPE}: Uses RoPE \cite{su2024roformer} for positional encoding, identical to pre-training.
    \item \textsc{NoPE}: Removes explicit positional encoding methods \cite{kazemnejad2023impact, wang-etal-2024-length}, i.e., RoPE is omitted during the continual pre-training.
    \item \textsc{R2N1}: A hybrid positional encoding method that interleaves RoPE and NoPE \cite{yang2025rope, meta2025llama}. For every three layers, the first two use RoPE while the last one uses NoPE.
    \item \textsc{N2R1}: The opposite of R2N1, i.e., every three layers, the first two use NoPE and the last one uses RoPE.
\end{itemize}



We train those models on 4 H100 GPUs for 50B tokens. We use the \texttt{allenai/olmes}\footnote{\url{https://github.com/allenai/olmes}} codebase for evaluation, which provides extensive test suites. We categorize our main evaluation tasks into three dimensions as follows:

\begin{itemize}%[leftmargin=*, nosep]
\item \textbf{Noisy Context} evaluates the model’s robustness to contexts containing large amounts of irrelevant information. Such ``noise’’ increases \textit{extraneous cognitive load} \cite{paas2003cognitive}, which can negatively affect problem-solving performance. We use the RULER benchmark \cite{hsiehruler}, which purposefully constructs contexts with irrelevant content for evaluation.


\item \textbf{Structured Data} evaluates performance on structured data such as tables and graphs. This setting highlights the importance of contextual organization, as linearizing such data into natural language often leads to significant structural information loss. We use NLGraph \cite{wang2023can} and HybridQA \cite{chen-etal-2020-hybridqa} to evaluate model performance on graph and table reasoning, respectively.


\item \textbf{Longer Context} evaluates model performance on sequences longer than those seen during training (i.e., 4K tokens), as positional encoding has been shown to strongly affect long-context generalization \cite{press2021train}. We use subsets of RULER and LongBench \cite{bai-etal-2024-longbench}, which contain examples with 4K to 16K tokens for this evaluation.

\end{itemize}
Notably, both the evaluations for noisy context and structured data are conducted within the training context length (i.e., 4K tokens). In contrast, the longer-context evaluation requires extended context lengths. To enable this, we apply the YaRN method to RoPE layers across all methods using the extrapolation hyperparameters in \citet{peng2024yarn}. More details are in Appendix~\ref{app:exp}.



\begin{table*}[]
\caption{Performance on LongBench. This uses F1 score for QA and fewshot tasks, while Rouge-L \cite{lin-2004-rouge} for the summarization tasks. We only evaluate on data that contains less than 16K tokens. Since the model is trained with a max of 4096 tokens, the YaRN method \cite{peng2024yarn} is used for all RoPE layers for context extrapolation. We use \textbf{bold} and \underline{underline} for best and second best results, respectively.\label{tab:lc_longbench}}
\resizebox{1.0\textwidth}{!}{
\begin{tabular}{lrrrrrrrrr}\toprule
\multirow{2}{*}{Method} & \multicolumn{2}{c}{Multidoc QA} & \multicolumn{2}{c}{Singledoc QA} & \multicolumn{2}{c}{Summarization} & \multicolumn{2}{c}{Fewshot} & \multirow{2}{*}{AVG.}\\
                        & 2WikiMultihopQA        & MuSiQue       & MultiFieldQA-en  & NarrativeQA  & GovReport        & QMSum        & TREC        & TriviaQA      \\\midrule
RoPE                    & 23.32           & 7.24          & 27.37             & 12.94        & 6.23               & 7.96         & 22.00       & 61.47  & 21.07        \\\hdashline
NoPE                    & 9.11            & 0.33          & 13.64             & 1.80         & 5.15               & 0.68         & 9.50        & 18.12 & 7.29\\
R2N1                    & \underline{25.88}           & \underline{7.31}          & \underline{31.28}             & \textbf{16.24}        & 4.53               & \underline{8.74}         & \underline{22.00}       & \underline{66.67}   & \underline{22.83}      \\
N2R1                    & 16.24           & 0.40          & 21.88             & 1.26         & \underline{8.74}               & 5.31         & 21.50       & 25.18    & 12.56     \\\hdashline
\implname                    & \textbf{30.86}           & \textbf{13.45}        & \textbf{33.12}             & \underline{15.24}        & \textbf{16.80}              & \textbf{12.53}        & \textbf{31.50}       & \textbf{73.02} & \textbf{28.31}\\ \bottomrule        
\end{tabular}
}
\end{table*}



\subsection{Results}

Experimental results show that \implname yields significant performance gains across all three evaluation dimensions. 

\paragraph{Noisy Context} For all evaluation subtasks, the input contexts are injected with irrelevant information. As shown in Table~\ref{tab:anti_noise}, even when all test contexts are within the training context length (i.e., 4K tokens), \implname outperforms RoPE by 11.04 points. This suggests that the re-positioning mechanism in \implname effectively reduces \textit{extraneous cognitive load} and improves robustness against noisy or distracting information in model inputs. 
 



\paragraph{Longer Context} The advantage of \implname becomes even more pronounced on long-context tasks. As shown in Figure~\ref{fig:lc_ruler}, \implname already surpasses all baselines at a context length of 4K. The performance gap further widens at 8K and 16K, which are lengths unseen during training. To rule out potential confounding effects from {\implname}’s noise robustness, we additionally evaluate on LongBench~\cite{bai-etal-2024-longbench}, which consists of more realistic long-context tasks. As shown in Table~\ref{tab:lc_longbench}, \implname consistently outperforms other baselines on LongBench by at least 5.48 points. In addition, the hybrid \textsc{R2N1} method, with interleaved RoPE and NoPE layers, achieves the second-best performance, consistent with the findings of \citet{yang2025rope}.




\paragraph{Structured Data} Since linearizing structured data (e.g., tables and graphs) into natural language can result in substantial loss of structural information, it is of interest to examine whether contextual re-positioning benefits such data types. As shown in Table~\ref{tab:struct}, \implname improves over the vanilla RoPE method by an average of 1.94 EM points. Interestingly, the NoPE method achieves the best performance on the graph dataset, suggesting that emphasizing local positional relationships may not be a valid assumption for graph-structured inputs.


\section{Analyses\label{sec:analyses}}
This section is to provide insights into the inner workings of \implname. To this end, we conducted detailed analyses to understand: 1) where the performance gains stem from; and 2) what patterns the positions assigned by \implname exhibit.

\begin{table}[!]
\centering
\small
\caption{Attention mass per token ($10^{-2}$) on different parts of context. We evaluate on NIAH task within the training context length (i.e., 4K tokens).\label{tab:attn_mass}} 
\resizebox{1.0\columnwidth}{!}{
\begin{tabular}{lccc}\toprule
Pos. Assignment     & Needle  & Query & Rest \\\midrule
Linear (e.g., RoPE) & 1.754  & 1.123 & 0.014   \\
Constant (e.g., NoPE) & 1.572  & 1.135 & 0.014 \\
\implname & 2.013  & 1.046 & 0.015  \\\bottomrule
\end{tabular}
}
\vspace{-1em}
\end{table}

\subsection{Attention Mass on Relevant Tokens\label{ssec:niah}} 
Since our \implname method re-organizes the context based on its intrinsic structure, we hypothesize that it can better capture long-distance dependencies by bringing distant but relevant tokens with closer positions.
To evaluate this effect, we analyze the attention patterns of methods with different types of position assignment strategies on the needle-in-a-haystack (NIAH) task \cite{kamradt2023needle, hsiehruler} and  quantitatively measure the attention mass, i.e., attention scores averaged across attention heads and layers, from generated tokens to three non-overlapping parts of the context, following \citet{yang2025rope}: 
\begin{itemize}%[leftmargin=*, nosep] 
\item \textbf{Needle}: tokens that correspond to the golden answer in the context. The ``needle'' tokens are generally distant from the generated tokens in the NIAH task.
\item \textbf{Query}: tokens that correspond to the user question and the continuation prefix in the context. Thus, they are closest to the generated tokens.
\item \textbf{Rest}: other tokens in the context. \end{itemize} 

We conduct our analysis on the NIAH dataset provided by RULER \cite{hsiehruler}, where the context follows the format: 

\begin{displayquote}
\texttt{Rest $\dots$ Needle $\dots$ Rest $\dots$ Query}
\end{displayquote}

As shown in Table~\ref{tab:attn_mass}, for needle tokens that are distant yet critical for generation, our \implname method allocates substantially more attention mass than both the linear (i.e., RoPE) and constant (i.e., NoPE) position assignment strategies. Compared with \implname, the linear position assignment also exhibits a stronger locality bias, encouraging attention allocation to nearby query tokens. In addition, the constant position assignment, which treats all positions uniformly, produces an attention pattern with much lower variance across the three parts. These findings explain how our \implname method achieves performance gains on tasks involving noisy context, and also support our motivation based on Cognitive Load Theory (CLT), where the germane load (e.g., the attention mechanism) can better process the context information with context re-positioning.

\subsection{Position Patterns Learned by \implname \label{ssec:pos_pattern}}
To better understand the patterns learned by \implname, we analyze the characteristics of the assigned positions, first focusing on their ranges and then on their local patterns.

We first collect statistics on the distances between the maximum and minimum assigned positions for each attention head:
$$
d^{k, h} = \mathrm{max}(\boldsymbol{z}^{k,h}) - \mathrm{min}(\boldsymbol{z}^{k,h}),
$$
where $\boldsymbol{z}^{k,h} = \{z_1^{k,h}, z_2^{k,h}, \dots, z_L^{k,h}\}$, $L$ is the number of tokens in input $\boldsymbol{x}$, and $k$ and $h$ represent the indices of the attention head and layer, respectively. 

We compare these statistics on a general benchmark (2K-token context) and the RULER benchmark (4K-token context). As shown in Figure \ref{fig:stats_pos_dist}, we find that \implname assigns larger positional distances $d$ on longer context lengths, but the largest distance  is still much smaller than the raw context length (i.e., 2K or 4K). This observation suggests that increasing the positional range to match the full input context length may not be necessary from the model’s perspective. Furthermore, the distribution of distances is non-uniform, unlike the linear positional assignment in RoPE. We hypothesize that assigning positions in a denser and non-linear continuous space contributes to improved performance on longer contexts in \S\ref{sec:exp}.



\begin{figure}[]
    \begin{center}
        \includegraphics[width=1.0\columnwidth]{figs/stats_pos_dist.png}
    \end{center}
    \caption{Statistics for the distances between maximum and minimum positions in each attention head of the LLM. The averaged and maximum number of tokens in the MMLUPro-Math benchmark are 1971 and 2512, while those for RULER-QA are 2995 and 3555, respectively. \label{fig:stats_pos_dist}}
\end{figure}



\begin{table*}[]
\centering
\caption{We use the default evaluation suites, including few-shot examples, prompts, and metrics, provided in \texttt{allenai/olmes} for evaluation. All evaluations are conducted within the training context length, i.e., 4096 tokens. We use \textbf{bold} and \underline{underline} to indicate the best and second-best results, respectively.\label{tab:general}}
\resizebox{1.0\textwidth}{!}{
\begin{tabular}{lcccccccccc}\toprule
Model  & ARC-C & ARC-E & BoolQ & CoQA & Drop & Hellaswag & MMLU-Pro & TriviaQA  & AVG.   \\\midrule
\textsc{RoPE}      & \textbf{47.99}                                   & \underline{75.25}                         & 72.12                 & 56.87            & 37.90            & \textbf{70.68}                         & \textbf{13.77}                                   & \textbf{54.98} & 53.70  \\\hdashline
\textsc{NoPE}      & 44.05                                   & 73.64                         & 66.70                 & 44.52            & 33.22            & 65.68                         & 10.70                                   & 43.43 & 47.74   \\
\textsc{R2N1} & 47.30                                   & \textbf{75.68}                         & \underline{73.04}                 & \textbf{59.31}            & \textbf{38.48}            & 69.26                         & 13.41                                   & 54.61 & \textbf{53.88}  \\
\textsc{N2R1} & 43.78                                   & 72.72                         & 69.38                 & 50.74            & 36.66            & 67.58                         & 11.99                                   & 47.31 & 50.02 \\\hdashline
\implname  & \underline{47.61}                                   & 74.87                         & \textbf{73.58}                 & \underline{57.44}            & \underline{38.17}            & \underline{70.08}                         & \underline{13.52}                                   & \underline{54.56}  & \underline{53.73} \\\bottomrule
\end{tabular}
}
\end{table*}


We then analyze the patterns of the positions assigned by \implname. We split the positions $\boldsymbol{z}^{l,h}$ into non-overlapping chunks with $\Delta$ tokens  $\{\boldsymbol{z}^{k,h}_{1:\Delta}, \boldsymbol{z}^{k,h}_{\Delta+1:2\Delta}, \dots, \boldsymbol{z}^{k,h}_{L-\Delta:L}\}$ and define three pattern types:
\begin{itemize}%[leftmargin=*, nosep]
    \item \textbf{Constant}: We calculate the average position value $a$ in the chunk. If all positions lie between $[a-\epsilon, a+\epsilon]$, we conjecture that the positions are close to a constant, indicating the pattern of NoPE that assigns all positions to a constant position. 
    \item \textbf{Mono}: If the positions in a chunk are monotonically increasing (i.e., $z_{i-1} < z_i < z_{i+1}$) or monotonically decreasing (i.e., $z_{i+1} < z_i < z_{i-1}$) for all $z_i$ in a chunk, we classify the pattern as \emph{monotonic}, similar to the position assignment strategy used by conventional position encoding methods.
    \item \textbf{Hybrid}: All other patterns, e.g., a mixture of constant and monotonic patterns.
\end{itemize}
We empirically set $\Delta=16$ and $\epsilon=0.2$ to provide insights for the learned patterns\footnote{Setting different values for $\Delta$ and $\epsilon$ results in different plots, but the overall conclusion still holds.}. As shown in Figure \ref{fig:stats_pos_pi}, we find that the \textbf{Mono} pattern is very rare (4\% of all chunks), and the model prefer the constant patterns (22\% of all chunks) than mono pattern.   The dominating pattern of positions assigned by \implname is \textbf{Hybrid}, indicating that the position patterns beneficial for LLMs are different from those pre-defined in previous work \cite{vaswani2017attention, su2024roformer}. This analysis is conducted on the RULER benchmark, but we find consistent observations on other benchmarks.

\begin{figure}[] \begin{center} \includegraphics[width=0.6\columnwidth]{figs/stats_pos_pi.png} \end{center} \caption{Statistics for the patterns of assigned positions. We split the context into non-overlapping chunks of 16 tokens. "Constant" means assigned positions are all close to a constant position, "Mono" means the positions are monotonically increasing or decreasing in the chunk, and "Hybrid" means all other patterns. \label{fig:stats_pos_pi}} \end{figure}


Besides the statistics of positions assigned by \implname, as shown in Appendix \ref{app:case_study}, we also conduct a case study to visualize the positions across different layers and attention heads of the LLM. We find that the assigned positions can capture the intrinsic structure of the input context, such as the segmentation of few-shot examples, which aligns with our CLT-based motivation.






\subsection{Performance on General Tasks}
As shown in Table \ref{tab:general}, along with the noticeable performance gain in previous experiments, our \implname method still achieves performance comparable to the \textsc{RoPE} method on extensive general benchmarks. This occurs even though changing from linear position assignment to \implname causes an inconsistency between pre-training and continued training. This observation indicates that our \implname, learned from general data, generalizes well to diverse types of data, even when questions in general benchmarks are typically short and precise, requiring almost no context reorganization.

\begin{table}
    \centering
    \caption{Efficiency comparison. FLOPs are reported for training on 50B tokens. Inference time (second) per token is evaluated using vLLM.\label{tab:efficency}}
    \begin{tabular}{lcc}\toprule
        Method & FLOPs & Dec. Time / Token \\\midrule
        \textsc{RoPE} & $3.84 \times 10^{20}$ & 0.0176 \\
        \implname & $3.87 \times 10^{20}$ & 0.0182   \\\bottomrule
    \end{tabular}
\end{table}



\subsection{Efficiency\label{app:efficiency}}
As shown in Table \ref{tab:efficency}, we compare the FLOPs and inference time between the vanilla model with \textsc{RoPE} method and \implname. We observe that \implname method is very lightweight, introducing only a $0.9\%$ increase in parameters while providing performance gains across many evaluation dimensions. When running inference for RULER benchmark \cite{hsiehruler} within training context length, the time cost of \implname is comparable to that of the vanilla model.

\section{Related Work}

The self-attention mechanism in Transformers \cite{vaswani2017attention} is inherently permutation-invariant, lacking an intrinsic understanding of input token order. To address the issue, a position encoding module is used in most Transformer-based models to map the position indices of input tokens into biases or embeddings that can be integrated into the model.


Bias-based position encoding methods, such as ALiBi \cite{press2021train}, KERPLE \cite{NEURIPS2022_37a41384}, and T5 \cite{raffel2020exploring}, integrate a distance bias directly into the attention logits to control the attention field of perception. Most position encoding methods, however, learn embeddings for position indices or distances. The absolute position embedding, which is added to the hidden states, has been widely used in Transformer-based models with small sequence lengths \cite{vaswani2017attention, radford2019language, devlin2019bert}. In the era of large language models (LLMs), the RoPE method \cite{su2024roformer}, which rotates queries and keys for calculating attention scores based on position distance, has become the \textit{de facto} choice for modern LLM architectures. Many subsequent variants have been proposed to improve the context extrapolation performance of RoPE \cite{peng2024yarn, chen2023extending, chen2024clex, chen-etal-2025-hope} or manually combining it with NoPE \cite{barbero2025round, yang2025rope}. Our work is orthogonal to these enhanced RoPE methods. Recently, \citet{li2025seqpe} proposed an alternative approach to model positional information in a sequential way. %Notably, many widely used position encoding methods are differentiable \cite{su2024roformer, vaswani2017attention, press2021train, li2025seqpe}.


In contrast to previous literature, our work focuses on position prediction for tokens within a given context, prior to the position encoding process. The most relevant work to ours is COPE \cite{golovneva2024contextual}, which learns an attention-logits-based gate module to predict a non-decreasing sequence for positions, aiming to segment high-level abstractions, such as sentences. However, The COPE method uses attention logits for gating, which incurs high time and memory costs due to the computation of $[B, L, L]$ tensors. It is also incompatible with other position encoding methods, like RoPE, and flash attention \cite{dao2024flashattention}, limiting its scalability. In contrast, our work focuses on re-organizing tokens within the context using a lightweight \implname module that is compatible with most position encoding methods.
Recently, \citet{zhang2025agentic} also highlighted the importance of input context and proposed ACE, which iteratively evolves the context in an agentic manner. Our work aims to enhance Transformer models with context re-positioning, which is orthogonal to ACE \cite{zhang2025agentic}.
\section{Conclusion and Discussion}
\label{sec:conclusion}

In this work, we revisit \mHC's design of residual connections from the perspective of stability and system portability. The iterative SK algorithm requires specialized kernels for efficient execution, creating engineering barrier for generic adoption. Moreover, through both theoretical analysis and empirical evaluation, we find that due to \mHC's reliance on a finite steps of SK iterations, its residual matrices may significantly deviate from doubly stochasticity, when the SK algorithm fails to converge, introducing potential risks of stability.  
To address these limitations, we propose \textbf{\mHC-lite}, a simple, strong, and efficient alternative to \mHC, achieved by re-parameterizing doubly stochastic matrices based on the Birkhoff–von Neumann theorem. The re-parameterization enables us to skip the SK iterations entirely, removing the approximation gap and supporting the computation with only basic operators, making our method a drop-in replacement for classical residual architectures, offering guaranteed robustness without sacrificing ease of deployment.

The design of \mHC-lite verifies a simple but powerful principle: exactness, when attainable, is often the most efficient form of approximation.
This shift from ``projection'' to ``reparameterization'' ensures the constraint hold by construction, eliminating approximation gaps (such as those induced by finitely many Sinkhorn--Knopp iterations) while enabling potentially more efficient implementations. 

\paragraph{On The Computational Efficiency of \mHC-lite for Larger $n$.}
An astute reader might notice that, although \mHC performs well when $n = 4$, its space and time complexity grow exponentially with $n$, raising potential concerns about the efficiency of this method when $n$ is larger. Here, we make two observations: 1) in the original HC paper~\cite{Zhu2024HyperConnections}, the authors conducted extensive ablation studies demonstrating that $n=4$ is indeed an superior choice in practice; 2) even if a larger $n$ is required, we can readily reduce the computational cost by sampling a subset of permutation matrices rather than including all of them. This is equivalent to restricting the feasible region to a subset of the Birkhoff polytope. The resulting residual matrix remains guaranteed to be doubly stochastic, while the computational budget can be tuned by controlling the number of sampled permutations.




% In the unusual situation where you want a paper to appear in the
% references without citing it in the main text, use \nocite
% \nocite{langley00}

\bibliography{example_paper}
\bibliographystyle{icml2026}

%%%%%%%%%%%%%%%%%%%%%%%%%%%%%%%%%%%%%%%%%%%%%%%%%%%%%%%%%%%%%%%%%%%%%%%%%%%%%%%
%%%%%%%%%%%%%%%%%%%%%%%%%%%%%%%%%%%%%%%%%%%%%%%%%%%%%%%%%%%%%%%%%%%%%%%%%%%%%%%
% APPENDIX
%%%%%%%%%%%%%%%%%%%%%%%%%%%%%%%%%%%%%%%%%%%%%%%%%%%%%%%%%%%%%%%%%%%%%%%%%%%%%%%
%%%%%%%%%%%%%%%%%%%%%%%%%%%%%%%%%%%%%%%%%%%%%%%%%%%%%%%%%%%%%%%%%%%%%%%%%%%%%%%
\newpage
\appendix
\onecolumn

\section{What’s the Real Difference between Conventional PEs, NoPE, and RePo?\label{app:comp}}

In the background section (\S\ref{sec:bg}), we use \textsc{RoPE} as a representative example to illustrate how conventional positional encoding methods rely on a strict linear pattern to assign positional information to tokens in the context.


Recently, researchers have found that the causal mask in the attention mechanism enables LLMs to implicitly learn positional information, and that removing explicit positional encoding can even achieve superior performance on structured data and long-context tasks. This approach is referred to as the NoPE method \cite{kazemnejad2023impact, yang2025rope, wang-etal-2024-length, barbero2025round}. We argue that the attention score of NoPE can be reformulated within the RoPE framework by assigning a constant positional value $a$:
\begin{align}
\mathbf{A}_{i,j}^{\text{NoPE}} &= \boldsymbol{q}_i^\top \boldsymbol{k}_j \nonumber \\
&= \boldsymbol{q}i^\top g\theta(0)\boldsymbol{k}_j \nonumber \\
&= \boldsymbol{q}i^\top g\theta(a - a)\boldsymbol{k}_j,
\end{align}
where $a$ denotes a uniform position value for all tokens, yielding a constant rotation matrix $g_\theta(0)$. Thus, under this reformulation, the key difference between RoPE and NoPE lies solely in how positions are assigned.

\begin{wraptable}{R}{0.4\columnwidth}
    \centering
    % \tiny
    \caption{Comparison between different methods. In RoPE-like methods, $g_\theta$ generates a rotation matrix based on a distance. The $j-i$ is the distance between $x_j$ and $x_i$, $g_\theta(0)$ is a constant rotation, and $z_j$ and $z_i$ are predicted positions by $f_\phi$ (Eq. \ref{eq:repo}).}
    \label{tab:method_comp}
        \begin{tabular}{lc}\toprule
        Method & Attention Score \\\midrule
        Linear (e.g., \textsc{RoPE}) & $\boldsymbol{q}_i^\top g_\theta(j-i)\boldsymbol{k}_j$ \\[0.5em]
        Constant (e.g.,\textsc{NoPE}) & $\boldsymbol{q}_i^\top g_\theta(0)\boldsymbol{k}_j$ \\[0.5em]
        \implname & $\boldsymbol{q}_i^\top g_\theta (z_j - z_i) \boldsymbol{k}_j$ \\\bottomrule
        \end{tabular}
\end{wraptable}


In addition, LLMs with interpolated  NoPE and RoPE layers \cite{yang2025rope, meta2025llama, barbero2025round} have been widely used architectures, which can be explained as hybrid position assignment strategies. In contrast to previous works that empirically configure the hybrid system with hyper-parameters, our
\implname shows higher expressiveness, as it can dynamically determine whether to adopt the conventional linear, NoPE-like constant, or hybrid position assignment for tokens in a given context. A comparison among the three approaches is summarized in Table~\ref{tab:method_comp}.
As explained in \S\ref{sec:bg},  when $z_j = z_i$, \implname effectively reduces to the NoPE pattern with the constant $z_j = z_i = a$. In contrast, if $z_j > z_i$ for $j > i$, it indicates that \implname adopts positional relationship similar to the conventional linear style, e.g., the strategt used in RoPE.
In our experiments and analyses, we will demonstrate that an LLM may dynamically select between constant and linear position assignments, or hybridize them with \implname module. Notably, although we use RoPE for the comparison, linear position assignment is widely adopted in conventional positional encoding methods \cite{vaswani2017attention, gehring2017convolutional, press2021train, li2025seqpe}, and our findings can be readily extended to these approaches.



\section{Details of Experiments\label{app:exp}}

\subsection{Extrapolation}
We use the following hyper-parameters to extend the context:
\begin{enumerate}
    \item 8K Tokens: \texttt{\{"rope\_type": "yarn", "factor": 2.0, "original\_max\_position\_embeddings": 4096\}}
    \item 16K Tokens: \texttt{\{"rope\_type": "yarn", "factor": 4.0, "original\_max\_position\_embeddings": 4096\}}
\end{enumerate}
We use the setting for ``16K Tokens'' for all the experiments on LongBench (Table \ref{tab:lc_longbench}).

\subsection{General Tasks}
We use the following task ids in \texttt{olmes} for the evaluation in Table \ref{tab:general}: \texttt{arc\_challenge:rc::large}, \texttt{arc\_easy:rc::olmes}, \texttt{boolq:rc::large}, \texttt{coqa::large}, \texttt{drop::large}, \texttt{hellaswag:rc::large}, \texttt{mmlu\_pro:cot::none}, \texttt{triviaqa::olmes}




\begin{wrapfigure}{R}{0.4\columnwidth}
    \centering
    % \includegraphics[width=\linewidth]{figs/ablation.pdf}
    \includegraphics[width=\linewidth]{figs/ablation.png}
    \caption{Sensitivity to the starting layer of \implname (i.e. $l=3,5,7$). We validate on the NIAH subtask of RULER benchmark and MMLUPro of general benchmarks.}
    \label{fig:ablation}
\end{wrapfigure}




\subsection{Ablation Study \label{app:ablation}}

As shown in Figure \ref{fig:ablation}, we evaluate the sensitivity of model performance to the starting layer of \implname, where $l=5$ indicates that \implname is applied beginning from the 5th layer of the LLM, while the vanilla \textsc{RoPE} is used for the lower layers. We conduct experiments on two subtasks, NIAH and MMLUPro. The results show that overall performance is robust to this hyperparameter. However, increasing $l$ slightly improves performance on general benchmarks while negatively affecting performance on NIAH. The results are consistent on other evaluation benchmarks.



\section{Preliminary Experiments \label{app:pre}}




\begin{figure*}[]
    \begin{center}
        \includegraphics[width=0.95\textwidth]{figs/layer_visual.png}
    \end{center}
    \caption{Visualization of predicted positions from a 4-layer GPT-2 model in the reversal task. The \textcolor{Cerulean}{area with blue background color} indicates input context, while the \textcolor{Apricot}{orange region} is the generated sequence. We use A-K to replace the real tokens to save space for illustration. The x and y-axis represent the input order and predicted position $z$ of a token, respectively.
    \label{fig:visual}}
\vspace{-1em}
\end{figure*}

In this experiment, we train a small-scale language model on a purposefully selected synthetic task, namely text reversal, to determine whether \implname can re-position tokens in the context.


In the text reversal task, a model is prompted to generate a given sequence of tokens $\boldsymbol{x} = \{x_1, x_2, \dots, x_L\}$ in a reversed order $\boldsymbol{x}^\prime = \{x_L, x_{L-1}, \dots, x_1\}$. \textit{Locality bias} does not apply here because the distance between a generated token and its corresponding dependent input token grows linearly as the generation proceeds. It is interesting to investigate whether the \implname method can learn beneficial re-positioning patterns from the task.

\subsection{Setup} We use the data and train/dev/test splits provided in \citet{kazemnejad2023impact} for the text reversal task. The sequence lengths $L$ in training are between $[2, 20]$, while we use the sequence lengths between $[2, 30]$ for evaluation.  The input sequence $\boldsymbol{x}$ is constructed from a fixed set of subwords that are shared across the three datasets, without regard to grammatical or semantic structure \cite{kazemnejad2023impact}.


We train a GPT-2 model with 4 layers for this task. We use NoPE, RoPE, and \implname\footnote{Notably, \implname functions solely as a position prediction module. For brevity, however, mentioning \implname implies the use of RoPE encoding together in this work.} methods to train the model. For the \implname method, we shared the parameters of $f_\phi$ for all the attention heads in each layer. All training hyper-parameters are set as in \citet{kazemnejad2023impact}. 


\begin{wrapfigure}{R}{0.5\columnwidth}
    \centering
    \includegraphics[width=\linewidth]{figs/synthetic_task.png}
    \caption{Performance on the text reversal task. We report the accuracy on all lengths of input sequences.}
    \label{fig:synthetic_task}
\end{wrapfigure}

\subsection{Findings}
As shown in Figure \ref{fig:synthetic_task}, due to the simplicity of the text reversal task, all the models achieve nearly perfect performance on short in-domain sequences ($L \leq 20$). However, when testing on examples with longer-range dependencies, i.e. $L > 20$, our \implname method demonstrates superior performance compared to both NoPE and RoPE. 

We further investigate the re-positioning patterns  learned by \implname that contribute to this performance gain. As illustrated in Figure \ref{fig:visual}, we visualize the predicted positions across different layers of the trained model and observe several intriguing patterns. The overall distribution of predicted positions is remarkably distinct from the pre-defined positional indices (e.g., 1 to 27). Specifically, we observe a mirror effect in layers 0, 2, and 3, where pairs of reversal tokens are assigned the same position indices. Additionally, we identify hybrid positional patterns across different parts of the context. For example, from layers 1 to 3, the model adopts a NoPE-like pattern for the opening tokens, assigining nearly identical position indices to tokens in the phrase “Reverse the following words:”, while exhibiting patterns with mirror symmetry for the input and output sequences. Notably, we did not introduce any inductive bias for this task; all patterns emerged in a purely data-driven manner. These intriguing patterns motivate us to further investigate \implname on more general datasets. 




\section{Case Study\label{app:case_study}}
As shown in Figure \ref{fig:case_study}, we visualize the positions assigned by \implname when testing on the MMLUPro benchmark \cite{wang2024mmlu} with few-shot examples. We observe that \implname learns distinct patterns across different layers and attention heads. Interestingly, as shown in Figure \ref{fig:case2}, the patterns of assigned positions roughly align with the semantic segmentation of the few-shot examples, demonstrating that \implname is capable of capturing the structure of the input context. Additionally, we find that some positions assigned by \implname are negative values, as shown in Figure \ref{fig:case3}. Those negative positions can be interpreted as rotations in a reversed direction under the RoPE framework. There also exist some outlier positions in the figures. Upon inspection, we find that they correspond to non-informative punctuation marks and function words, such as ``.'' and ``such.''

\begin{figure}[htbp]
    \centering
    % --- First Subfigure ---
    \begin{subfigure}{\linewidth}
        \centering
        \includegraphics[width=0.9\linewidth]{figs/l5h1.png}
        \caption{Layer 5 \& Attention Head 1}
        \label{fig:case1}
    \end{subfigure}
    
    \vspace{1em} % Add vertical space between the images
    
    % --- Second Subfigure ---
    \begin{subfigure}{\linewidth}
    \centering
        \includegraphics[width=0.9\linewidth]{figs/l8h0.png}
        \caption{Layer 8 \& Attention Head 0}
        \label{fig:case2}
    \end{subfigure}


    \begin{subfigure}{\linewidth}
    \centering
        \includegraphics[width=0.9\linewidth]{figs/l13h3.png}
        \caption{Layer 13 \& Attention Head 3}
        \label{fig:case3}
    \end{subfigure}


    \caption{Visualization of positions assigned by \implname. The \implname is continuously trained on general data. The visualization data is from MMLUPro with few-shot examples. Symbols in \textcolor{orange}{orange} belong to the prompt, while symbols in \textcolor{blue}{blue} and \textcolor{red}{red} represent questions and answers in few-shot examples.}
    \label{fig:case_study}
\end{figure}

%%%%%%%%%%%%%%%%%%%%%%%%%%%%%%%%%%%%%%%%%%%%%%%%%%%%%%%%%%%%%%%%%%%%%%%%%%%%%%%
%%%%%%%%%%%%%%%%%%%%%%%%%%%%%%%%%%%%%%%%%%%%%%%%%%%%%%%%%%%%%%%%%%%%%%%%%%%%%%%

\end{document}

% This document was modified from the file originally made available by
% Pat Langley and Andrea Danyluk for ICML-2K. This version was created
% by Iain Murray in 2018, and modified by Alexandre Bouchard in
% 2019 and 2021 and by Csaba Szepesvari, Gang Niu and Sivan Sabato in 2022.
% Modified again in 2023 and 2024 by Sivan Sabato and Jonathan Scarlett.
% Previous contributors include Dan Roy, Lise Getoor and Tobias
% Scheffer, which was slightly modified from the 2010 version by
% Thorsten Joachims & Johannes Fuernkranz, slightly modified from the
% 2009 version by Kiri Wagstaff and Sam Roweis's 2008 version, which is
% slightly modified from Prasad Tadepalli's 2007 version which is a
% lightly changed version of the previous year's version by Andrew
% Moore, which was in turn edited from those of Kristian Kersting and
% Codrina Lauth. Alex Smola contributed to the algorithmic style files.
