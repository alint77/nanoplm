\begin{abstract}
In-context learning is fundamental to modern Large Language Models (LLMs); however, prevailing architectures impose a rigid and fixed contextual structure by assigning linear or constant positional indices.
Drawing on Cognitive Load Theory (CLT), we argue that this uninformative structure increases extraneous cognitive load, consuming finite working memory capacity that should be allocated to deep reasoning and attention allocation.
To address this, we propose \implname, a novel mechanism that reduces extraneous load via context re-positioning. Unlike standard approaches, \implname utilizes a differentiable module, $f_\phi$, to assign token positions that capture contextual dependencies, rather than replying on pre-defined integer range. By continually pre-training on the OLMo-2 1B backbone, we demonstrate that \implname significantly enhances performance on tasks involving noisy contexts, structured data, and longer context length, while maintaining competitive performance on general short-context tasks. Detailed analysis reveals that \implname successfully allocate higher attention to distant but relevant information, assign positions in dense and non-linear space, and capture the intrinsic structure of the input context. 
Our code is available at \url{https://github.com/SakanaAI/repo}.
\end{abstract}