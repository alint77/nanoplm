%%%%%%%% ICML 2026 EXAMPLE LATEX SUBMISSION FILE %%%%%%%%%%%%%%%%%

\documentclass{article}

% Recommended, but optional, packages for figures and better typesetting:
\usepackage{microtype}
\usepackage{graphicx}
\usepackage{subcaption}
\usepackage{booktabs} % for professional tables
\usepackage{amsmath} % 导言区
\usepackage{makecell}
\usepackage{xcolor}
\usepackage{xspace}


% hyperref makes hyperlinks in the resulting PDF.
% If your build breaks (sometimes temporarily if a hyperlink spans a page)
% please comment out the following usepackage line and replace
% \usepackage{icml2026} with \usepackage[nohyperref]{icml2026} above.
\usepackage{hyperref}


% Attempt to make hyperref and algorithmic work together better:
\newcommand{\theHalgorithm}{\arabic{algorithm}}
\newcommand{\todo}{\color{green}}
\newcommand{\gjy}{\color{red}}
\newcommand{\yyy}[1]{{\color{orange}{[yyy: #1]}}}
\newcommand{\mHC}{\textit{m}HC\xspace}
% Use the following line for the initial blind version submitted for review:
% \usepackage{icml2026}

% For preprint, use
\usepackage[preprint]{icml2026}

% If accepted, instead use the following line for the camera-ready submission:
% \usepackage[accepted]{icml2026}

\usepackage{amsmath}
\usepackage{amssymb}
\usepackage{mathtools}
\usepackage{amsthm}


% if you use cleveref..
\usepackage[capitalize,noabbrev]{cleveref}

%%%%%%%%%%%%%%%%%%%%%%%%%%%%%%%%
% THEOREMS
%%%%%%%%%%%%%%%%%%%%%%%%%%%%%%%%
\theoremstyle{plain}
\newtheorem{theorem}{Theorem}[section]
\newtheorem{proposition}[theorem]{Proposition}
\newtheorem{lemma}[theorem]{Lemma}
\newtheorem{corollary}[theorem]{Corollary}
\theoremstyle{definition}
\newtheorem{definition}[theorem]{Definition}
\newtheorem{assumption}[theorem]{Assumption}
\theoremstyle{remark}
\newtheorem{remark}[theorem]{Remark}

% Todonotes is useful during development; simply uncomment the next line
%    and comment out the line below the next line to turn off comments
%\usepackage[disable,textsize=tiny]{todonotes}
% \usepackage[textsize=tiny]{todonotes}

% The \icmltitle you define below is probably too long as a header.
% Therefore, a short form for the running title is supplied here:
\icmltitlerunning{\textit{m}HC-lite: You Don't Need 20 Sinkhorn-Knopp Iterations}

\begin{document}

\twocolumn[
  \icmltitle{\textit{m}HC-lite: You Don't Need 20 Sinkhorn-Knopp Iterations}

  % It is OKAY to include author information, even for blind submissions: the
  % style file will automatically remove it for you unless you've provided
  % the [accepted] option to the icml2026 package.

  % List of affiliations: The first argument should be a (short) identifier you
  % will use later to specify author affiliations Academic affiliations
  % should list Department, University, City, Region, Country Industry
  % affiliations should list Company, City, Region, Country

  % You can specify symbols, otherwise they are numbered in order. Ideally, you
  % should not use this facility. Affiliations will be numbered in order of
  % appearance and this is the preferred way.
  \icmlsetsymbol{equal}{*}

  \begin{icmlauthorlist}
    \icmlauthor{Yongyi Yang}{equal,umich,harvard,ntt}
    \icmlauthor{Jianyang Gao}{equal,ntu}
  \end{icmlauthorlist}

  \icmlaffiliation{umich}{CSE, University of Michigan, Ann Arbor, USA}
  \icmlaffiliation{ntu}{CCDS, Nanyang Technological University, Singapore}
  \icmlaffiliation{harvard}{CBS-NTT Physics of Intelligence Program, Harvard University, Cambridge, USA}
  \icmlaffiliation{ntt}{Physics of Artificial Intelligence Group, NTT Research Inc, USA}

  \icmlcorrespondingauthor{Yongyi Yang}{yongyi@umich.edu}
  \icmlcorrespondingauthor{Jianyang Gao}{jianyang.gao@ntu.edu.sg}

  % You may provide any keywords that you find helpful for describing your
  % paper; these are used to populate the "keywords" metadata in the PDF but
  % will not be shown in the document
  \icmlkeywords{Machine Learning, ICML}

  \vskip 0.3in
]

% this must go after the closing bracket ] following \twocolumn[ ...

% This command actually creates the footnote in the first column listing the
% affiliations and the copyright notice. The command takes one argument, which
% is text to display at the start of the footnote. The \icmlEqualContribution
% command is standard text for equal contribution. Remove it (just {}) if you
% do not need this facility.

% Use ONE of the following lines. DO NOT remove the command.
% If you have no special notice, KEEP empty braces:
\printAffiliationsAndNotice{}  % no special notice (required even if empty)
% Or, if applicable, use the standard equal contribution text:
% \printAffiliationsAndNotice{\icmlEqualContribution}

\begin{abstract}
Hyper-Connections (HC) generalizes residual connections by introducing dynamic residual matrices that mix information across multiple residual streams, accelerating convergence in deep neural networks. However, unconstrained residual matrices can compromise training stability. To address this, DeepSeek's Manifold-Constrained Hyper-Connections (\mHC) approximately projects these matrices onto the Birkhoff polytope via iterative Sinkhorn--Knopp (SK) normalization. We identify two limitations of this approach: (i) finite SK iterations do not guarantee exact doubly stochasticity, leaving an approximation gap that can accumulate through network depth and undermine stability; (ii) efficient SK implementation requires highly specialized CUDA kernels, raising engineering barriers and reducing portability. Motivated by the Birkhoff--von Neumann theorem, we propose \textbf{\mHC-lite}, a simple reparameterization that explicitly constructs doubly stochastic matrices as convex combinations of permutation matrices. This approach guarantees exact doubly stochasticity by construction and can be implemented using only native matrix operations. Extensive experiments demonstrate that \mHC-lite matches or exceeds \mHC in performance while achieving higher training throughput with a naive implementation and eliminating the residual instabilities observed in both HC and \mHC. The code is publicly available at \url{https://github.com/FFTYYY/mhc-lite}.
\end{abstract}

\section{Introduction}


\begin{figure}[]
    \begin{center}
        \includegraphics[width=0.95\columnwidth]{figs/repo_overall.png}
    \end{center}
    \caption{Overall performance on four evaluation dimensions.\label{fig:overall}}
\end{figure}

The emergence of large language models (LLMs) \cite{brown2020language} has enabled a wide range of applications, including few-shot learning \cite{brown2020language}, retrieval-augmented generation \cite{lewis2020retrieval, yao2023react}, and agentic systems \cite{schick2023toolformer, park2023generative}. At the core of these applications is in-context learning \cite{shen2024position}, a form analogous to human \emph{working memory}, where information within a limited context window is temporarily stored and processed to solve a task. Consequently, exploring how to effectively utilize context information has become a fundamental research line in the LLM era \cite{wei2022chain, weston2023system, chen2023extending}.

Recent studies show that an LLM’s ability to leverage contextual information is strongly influenced by its position encoding scheme \cite{vaswani2017attention, press2021train, su2024roformer}. Most LLMs impose a fixed contextual structure by assigning tokens consecutive integer indices from $0$ to $L-1$ \cite{vaswani2017attention} or a constant index $a$ for all tokens \cite{kazemnejad2023impact}. These indices are then integrated into a model through position encoding functions, enforcing a rigid organization of context.


Although fixed position assignments have become the \emph{de facto} standard, they deviate from how human working memory processes information. According to Cognitive Load Theory (CLT), the capacity of working memory during problem solving can be consumed by costs arising from how information is organized and presented, referred to as \textit{extraneous load} \cite{sweller1994cognitive, paas2003cognitive}. CLT studies  suggest that humans can actively reduce this extraneous load by reorganizing context, e.g., grouping related information into meaningful chunks \cite{miller1956magical} or removing irrelevant details from instructions \cite{sweller1994cognitive}. Since working memory capacity is fixed, the capacity saved through such reorganization can be reallocated to deeper reasoning processes associated with the \textit{the germane load}, thereby improving problem-solving performance \cite{sweller1994cognitive}.

However, the critical ability to reorganize and restructure contextual information \cite{vaswani2017attention, yang2025qwen3, dubey2024llama} is absent from the architectural design of modern LLMs. From the perspective of CLT \cite{sweller1994cognitive, paas2003cognitive}, rigid linear or constant position structures can be interpreted as introducing additional extraneous load, which in turn impairs attention allocation and deeper contextual reasoning (i.e., germane processing). As a consequence, tasks that require strong long-range or fine-grained contextual dependencies, e.g., needle-in-a-haystack (NIAH) problems \cite{kamradt2023needle} or question answering under highly diluted contexts \cite{hsiehruler}, exhibit notable performance degradation, mirroring the effects predicted by CLT under high extraneous load. Moreover, from a probabilistic standpoint, these position assignment strategies, essentially uniform distributions over fixed integer ranges,  are
the least informative organizations of context and therefore limit representational efficiency.   


We propose an internal mechanism for LLMs to reduce \textit{extraneous load} by re-organizing the positions of tokens.
% , thereby conserving the finite working memory capacity for beneficial \textit{germane load}. 
We formalize this process, termed \emph{context Re-Positioning} (\implname), as learning non-constrained position values based on information relevance of tokens, instead of using the fixed linear positions in prior work. To this end, we introduce a differentiable module $f_\phi$, which assigns a position value in continuous space for each token based on its hidden state. The $f_\phi$ can be independently learned for each attention head of an LLM. Trained on general data, $f_\phi$ learns to re-position tokens free from conventional constraints like monotonicity or integer values. The continuity of modern position encoding functions, e.g., RoPE \cite{su2024roformer} and ALiBi \cite{press2021train} methods, is key to the end-to-end optimization of $f_\phi$, as it allows these assigned positions to be integrated in a differentiable manner. 


We find that LLMs using the \implname method achieve consistent performance gains on tasks involving noisy context, structured data, and longer context. In our experiments, we continually pre-trained LLMs with the \implname method and several baselines based on the OLMo-2 1B model.
% \cite{walsh2} as the backbone for 50B tokens. 
Within the training context length, our \implname method outperforms other baselines by at least 6.24 and 1.16 points on noisy context and structured data tasks, respectively. In addition, when extending the testing context length to 16K tokens using the YaRN \cite{peng2024yarn} method, our \implname method outperforms other baselines by at least 13.25 EM points on the QA and Needle-in-the-haystack (NIAH) tasks and 5.48 points on LongBench \cite{bai-etal-2024-longbench}. Alongside these performance gains, \implname achieves comparable results on extensive general benchmarks \cite{wang2024mmlu, clark2018think, zellers-etal-2019-hellaswag}, which are typically short and require little reorganization.

To better understand whether our method aligns with the CLT-based motivation and where the performance gains of the \implname method come from, we conducted a series of detailed analyses. \textbf{First}, to evaluate how different methods handle long-range dependencies, we compared the attention distributions across methods, particularly focusing on their treatment of distant but relevant information. In the NIAH task (Section~\ref{ssec:niah}), we observed that \implname allocates more attention to the most critical ``needle'' tokens compared to the baseline methods, while directing less attention to the nearest ``query'' tokens. This behavior breaks the typical locality bias \cite{yang2025rope, press2021train}, dynamically adjusting based on the context. \textbf{Second}, the positions assigned by \implname exist in a denser, more non-linear space, which is critical for enhancing its generalization to longer contexts, such as extending from 4K to 16K tokens. \textbf{Finally}, an interesting finding is that \implname learns position patterns that resemble a hybrid of previous position assignment strategies, such as assigning constant positions $a \pm \epsilon$ \cite{kazemnejad2023impact} or enforcing a monotonic position sequence \cite{vaswani2017attention, yang2025rope, press2021train}, within a given context span (Section~\ref{ssec:pos_pattern}). In our case study (Appendix~\ref{app:case_study}), we also found that the positions assigned by \implname capture the intrinsic structure of the input context (e.g., segmentation of few-shot examples). 


\section{Background}
\label{sec:background}

\label{sec:background:transformer_residual}

The residual connection paradigm, originally introduced by ResNet~\citep{he2016deep}, has been serving as the fundamental backbone of modern deep learning. It builds an identity mapping path that mitigates the vanishing gradient problem and enables the training of extremely deep networks~\citep{he2016identity}. This design was subsequently adopted by the Transformer architecture~\citep{vaswani2017attention} and has proven essential for the scalability of large language models (LLMs), such as GPT-3~\citep{brown2020language} and Llama~\citep{touvron2023llama}.

Despite its widespread success, the standard residual connection has inherent limitations. The single-stream design restricts information flow to a single pathway, potentially limiting the representational capacity of very deep networks~\citep{huang2017densely}. Moreover, the fixed identity mapping, while stabilizing training, offers no adaptability to the varying computational demands across different layers or input contexts~\citep{srivastava2015highway}. These observations have motivated recent research into more flexible and expressive connection mechanisms that go beyond the simple identity shortcut while preserving training stability~\citep{xie2024residual,Zhu2024HyperConnections,mak2025residual,bhendawade2025mr,xie25mhc,liu2025th}.

\paragraph{Hyper-Connections (HC).}
Hyper-Connections (HC) generalizes residual connections by expanding a single residual stream into multiple streams and introducing dynamic connections among these streams~\cite{Zhu2024HyperConnections}. This generalized residual connection enriches the model’s connectivity and has been reported to accelerate convergence with little additional computation~\cite{Zhu2024HyperConnections}. 
Let $\boldsymbol{x}_l\in \mathbb{R}^{n\times C}$ denote the input feature of the $l$-th layer, where $n$ is the number of residual streams and $C$ is the dimensionality. The architecture is formulated as follows.
\begin{align}
    \boldsymbol{x}_{l+1} = {\boldsymbol{H}}^{\text{res}}_l \boldsymbol{x}_l + {\boldsymbol{H}}^{\text{post}}_l f({\boldsymbol{H}}^{\text{pre}}_l\boldsymbol{x}_l;\mathcal{W}_l) \label{eq: hc}
\end{align}
where the residual matrix ${\boldsymbol{H}}^{\text{res}}_l\in \mathbb{R}^{n\times n}$ is dynamically determined by learnable parameters and $\boldsymbol{x}_l$, and is used to mix the residual streams. The terms ${\boldsymbol{H}}^{\text{pre}}_l, {\boldsymbol{H}}^{\text{post}}_l\in \mathbb{R}^{1\times n}$ are decided by learnable parameters and $\boldsymbol{x}_l$, and is used to aggregate the input and expand the output respectively. The term $f(\cdot ;\mathcal{W}_l)$ represents a learnable function parameterized by weights $\mathcal{W}_l$. 
For the detailed computation of ${\boldsymbol{H}}^{\text{\text{res}}}_l$ and ${\boldsymbol{H}}^{\text{\text{pre}}}_l, {\boldsymbol{H}}^{\text{\text{post}}}_l$ in HC, we refer readers to the original paper~\cite{Zhu2024HyperConnections}. 

\paragraph{Manifold-Constrained Hyper-Connections (\mHC).}
Manifold-Constrained Hyper-Connections modifies the computation of ${\boldsymbol{H}}^{\text{\text{pre}}}_l,{\boldsymbol{H}}^{\text{\text{post}}}_l$ and ${\boldsymbol{H}}^{\text{\text{res}}}_l$, particularly, attempting to constrain ${\boldsymbol{H}}^{\text{\text{res}}}_l$ on the Birkhoff polytope $\mathcal{B}_n$, i.e., the set of doubly stochastic matrices, whose definition is as follows. 
\label{eq:birkhoff_polytope}
\begin{align}
    \mathcal{B}_n=
\left\{
\boldsymbol{X} \in \mathbb{R}^{n\times n} \;\middle|\;
\boldsymbol{X}^\top\mathbf{1}_n=\boldsymbol{X}\mathbf{1}_n = \mathbf{1}_n,\;
\boldsymbol{X} \ge 0 \notag
\right\}
\end{align}
where $\mathbf{1}_n$ denotes the all-ones vector and $\boldsymbol{X}\ge 0$ is entrywise.
The doubly stochastic matrices exhibit identity-like stability because their spectral norms are bounded by 1 and the set is closed under matrix multiplication: repeated composition of doubly stochastic matrices is still doubly stochastic.
Let $\boldsymbol{x}_l\in \mathbb{R}^{n\times C}$ denote the input feature in the $l$-th layer and $\boldsymbol{\hat x}_l\in \mathbb{R}^{1\times nC}$ denote the flatten input feature. The computation of \mHC is detailed as follows.
\begin{align}
    {\boldsymbol{\hat x}_l'} &= \mathop{ \mathrm{ RMSNorm } } (\boldsymbol{\hat x}_l) \notag 
    \\ {\boldsymbol{H}}^{\text{\text{pre}}}_l &= \mathop{ \mathrm{ sigmoid } }
    \left(
    {\alpha^{\text{pre}}_l}
    {\boldsymbol{\hat x}_l'}\boldsymbol{W}^{\text{\text{pre}}}_l + \boldsymbol{b}_l^{\text{pre}}\right) \notag
    \\ {\boldsymbol{H}}^{\text{\text{post}}}_l &= 2\cdot \mathop{ \mathrm{ sigmoid } }\left({\alpha^{\text{post}}_l}{\boldsymbol{\hat x}_l'}\boldsymbol{W}^{\text{\text{post}}}_l + \boldsymbol{b}_l^{\text{post}}\right) \notag
    \\ {\boldsymbol{H}}^{\text{\text{res}}}_l &=\mathop{ \mathrm{ SK } }\left(\exp\left(\mathop{ \mathrm{ mat } }\left({\alpha^{\text{res}}_l}{\boldsymbol{\hat x}_l'}\boldsymbol{W}^{\text{\text{res}}}_l + \boldsymbol{b}_l^{\text{res}}\right)\right)\right) \label{eq: skexp}
\end{align}
where $\boldsymbol{W}^{\text{\text{pre}}}_l,\boldsymbol{W}^{\text{\text{post}}}_l\in \mathbb{R}^{nC\times n}$ and $\boldsymbol{W}^{\text{\text{res}}}_l\in \mathbb{R}^{nC\times n^2}$ are learnable weight matrices in the $l$-th layer. The terms $\boldsymbol{b}^{\text{pre}}_l, \boldsymbol{b}^{\text{post}}_l\in \mathbb{R}^{1\times n}$ and $\boldsymbol{b}^{\text{res}}_l\in \mathbb{R}^{1\times n^2}$ are learnable biases. The terms ${\alpha^{\text{pre}}_l},{\alpha^{\text{post}}_l}$ and ${\alpha^{\text{res}}_l}$ are learnable scalars. 
The function $\mathop{ \mathrm{ mat } }(\cdot)$ reshapes a matrix from $\mathbb{R}^{1\times n^2}$ to $\mathbb{R}^{n\times n}$.
The $\mathop{ \mathrm{ RMSNorm } }(\cdot)$ refers to the RMSNorm~\cite{rmsnorm}.
The $\exp(\cdot)$ function is entrywise. The $\mathop{ \mathrm{ SK } }(\cdot)$ iteration alternately rescales all columns and rows so that their sums equal 1. In the setup of \mHC, the SK iteration is repeated 20 times. 


% \input{content/3-preliminaries}


\section{Methodology}
\label{sec: method}
As discussed in Section~\ref{sec: intro}, \mHC's reliance on a finite number of SK iterations raises concerns regarding portability and stability. 
From a system perspective, achieving competitive efficiency for SK iterations typically relies on specialized, fused CUDA kernels, making this component difficult to serve as a drop-in replacement for standard residual connections across different frameworks.
Beyond portability, a more fundamental issue lies in the stability of the residual matrices.
In particular, finite-step approximation can lead to non-negligible deviations from exact doubly stochasticity, which may accumulate across depth and undermine the stability that \mHC aims to achieve.
We analyze this stability issue in detail in Section~\ref{subsec: stablity}.
These observations together motivate a re-parameterization in Section~\ref{subsec: mhc-lite}, which ensures exact doubly stochasticity by construction and avoids heavy customization of CUDA kernels.

\subsection{Analysis of the Stability}
\label{subsec: stablity}
In \mHC, a fixed number of SK iterations (e.g., 20 iterations in \mHC) does not guarantee a high-quality approximation when the convergence is slow.
Classical studies on matrix scaling show that SK is not uniformly fast in general~\cite{lsw,knight2008sinkhorn,Chakrabarty21}.
For general nonnegative matrices, the SK algorithm only comes with a worst-case iteration bound as follows: to obtain an approximation of doubly stochasticity whose $\ell_1$-error~\footnote{This bound follows from Corollary 2 in~\cite{Chakrabarty21}. Here, the $\ell_1$-error indicates the summation of the errors of all the column/raw sums, i.e., $\ell_1\text{-error}({\boldsymbol{X} })
:= \|\boldsymbol{X}\mathbf 1_n - \mathbf 1_n\|_{\ell_1}
 + \|\boldsymbol{X}^{\top}\mathbf 1_n - \mathbf 1_n\|_{\ell_1}$.} is at most $\epsilon$, it may require up to $O\left(\frac{n^2\log (n/\nu)}{\epsilon ^2}\right)$ iterations, where the relative range $\nu$ is defined by
\begin{align}
\nu := \frac{\displaystyle\min_{i,j:\,x_{i,j} > 0} x_{i,j}}{\displaystyle\max_{i,j} x_{i,j}},\label{eq:nu}
\end{align}
where $x_{i,j}$ is the $(i,j)$-th entry of ${\boldsymbol{X}}$. Even for strictly positive matrices, convergence remains sensitive to $1/\nu$ and can be extremely slow when $1/\nu$ is large~\cite{lsw} (see the example in Section~\ref{sec: intro}).

This issue is practically relevant in \mHC.
As shown in Equation~(\ref{eq: skexp}), the SK input is obtained by exponentiating an affine function of the features, which can yield ill-conditioned matrices with very large relative range.
In our measurements (Figure~\ref{fig:nu}), approximately $27.9\%$ of SK inputs satisfy $1/\nu \ge 10^{13}$.
Under such inputs, a fixed SK budget may fail to produce a near-doubly-stochastic matrix.
Figure~\ref{fig:h_res} shows that the column sum of a single residual matrix in \mHC may deviate from $1$ by up to $100\%$. 
More importantly, these per-layer deviations can accumulate through depth: Figure~\ref{fig:h_res} shows that the column sums of $\prod_l \boldsymbol{H}^{\text{res}}_l$ may deviate from $1$ by up to $220\%$ in a $24$-layer network, implying the risks of instability when models further scale up. In practice, a latest model constructs a 1{,}000-layer network for self-supervised reinforcement learning~\cite{wang2025} based on the classical identity residual connection~\cite{he2016deep}.
This empirical trend indicates the importance of stable residual matrices with theoretical guarantees. 


\begin{figure*}[th]
\begin{minipage}{0.49\linewidth}
    \centering
    \includegraphics[width=1\linewidth]{figures/large_fineweb_gnorm.pdf}
\end{minipage}
\begin{minipage}{0.49\linewidth}
    \centering
    \includegraphics[width=1\linewidth]{figures/large_fineweb_gnorm_avg.pdf}
    
\end{minipage}
\caption{\textbf{Gradient-norm dynamics during training.} We compare the evolution of gradient norms over the course of training. \textbf{Left:} overall trajectories, showing that both \mHC and \mHC-lite exhibit substantially smaller gradient norms (and improved stability) than HC. \textbf{Right:} a zoomed-in view of \mHC and \mHC-lite; curves are smoothed using a 200-step moving average, and the shaded region indicates the standard deviation within the same window. From the zoomed-in view, it is clear that \mHC-lite yields a smaller mean gradient norm and reduced fluctuations compared to \mHC. Results are obtained with the \textsf{L} model on the \texttt{FineWeb-Edu} dataset.}
    \label{fig:grad-norm}
\end{figure*}
\begin{table*}[thbp]
\centering
{\setlength{\tabcolsep}{4pt}\begin{tabular}{lcc cc cc  cc cc cc}
\toprule
Dataset
& \multicolumn{6}{c}{\texttt{OpenWebText}}
& \multicolumn{6}{c}{\texttt{FineWeb-Edu}} \\
\cmidrule(lr){2-7}\cmidrule(lr){8-13}
Model Scale
& \multicolumn{2}{c}{\sf{S}} & \multicolumn{2}{c}{\sf{M}} & \multicolumn{2}{c}{\sf{L}}
& \multicolumn{2}{c}{\sf{S}} & \multicolumn{2}{c}{\sf{M}} & \multicolumn{2}{c}{\sf{L}} \\
\cmidrule(lr){2-3}\cmidrule(lr){4-5}\cmidrule(lr){6-7}
\cmidrule(lr){8-9}\cmidrule(lr){10-11}\cmidrule(lr){12-13}
& Train & Val & Train & Val & Train & Val
& Train & Val & Train & Val & Train & Val \\
\midrule
Residual & 3.566 & 3.562 & 3.343 & 3.336 & 3.237 & 3.242 & 3.526 & 3.536 & 3.316 & 3.321 & 3.238 & 3.240 \\
HC       & 3.475 & 3.471 & 3.272 & 3.264 & 3.244 & 3.248 & 3.463 & \textbf{3.473} & 3.266 & 3.273 & 3.241 & 3.244 \\
\mHC      & 3.474 & 3.469 & 3.267 & 3.259 & \textbf{3.191} & \textbf{3.198} &  \textbf{3.462} & \textbf{3.473} & \textbf{3.237} & \textbf{3.243} & 3.200 & 3.204  \\
\mHC-lite & \textbf{3.471} & \textbf{3.467} & \textbf{3.261} & 
\textbf{3.255} & 3.194 & \textbf{3.198} & 3.468 & 3.477 & 3.243 & 3.249 & \textbf{3.181} & \textbf{3.185}  \\
\bottomrule
\end{tabular}}
\vspace{0.5em}
\caption{\textbf{Loss of trained models.} We report training and validation loss at the end of training. To mitigate stochastic fluctuations, training loss is computed as a moving average over the last 200 iterations.}\label{tab:loss}
\end{table*}

\subsection{Re-parameterization and \mHC-lite}
\label{subsec: mhc-lite}
Our methodology is based on the Birkhoff-von Neumann Theorem~\cite{Birkhoff,Neumann53}, which is also highlighted by \mHC~\cite{xie25mhc}.
To keep the paper self-contained, we restate the theorem as follows.
\begin{theorem}[The Birkhoff-von Neumann theorem]
For any $\boldsymbol{X}\in\mathcal{B}_n$, there exists a weight $\mathbf{a}=(a_1,...,a_{n!}) \in \mathbb{R}^{1\times n!}$, where $a_k \ge 0,\forall k\in [n!], \|\mathbf{a}\|_{\ell_1}=1$, such that 
$$\boldsymbol{X} = \sum_{i=1}^{n!}a_k \boldsymbol{P}_k$$
where $\left\{\boldsymbol{P}_k\right\}_{k=1}^{n!}$ is the sequence of $n\times n$ permutation matrices. 
\end{theorem}

Based on the Birkhoff-von Neumann theorem, we directly represent doubly stochastic matrices as convex combinations of permutation matrices. 
This parameterization guarantees that the matrix is precise doubly stochastic. Furthermore, by eliminating iterative approximations, the parameterization removes their computational overhead in both training and inferencing, avoiding the heavy reliance of highly specialized infrastructures.


In \mHC-lite, to control for confounding factors, we keep the structure of \mHC unchanged, except for ${\boldsymbol{H}}^{\text{\text{res}}}_l$.
Let $\boldsymbol{x}_l\in \mathbb{R}^{n\times C}$ denote the input feature in the $l$-th layer and $\boldsymbol{\hat x}_l\in \mathbb{R}^{1\times nC}$ denote the flatten input feature. Then we build mappings ${\boldsymbol{H}}^{\text{\text{res}}}_l,{\boldsymbol{H}}^{\text{\text{pre}}}_l$ and ${\boldsymbol{H}}^{\text{\text{post}}}_l$ dynamically based on $\boldsymbol{x}_l$ as follows. 
\begin{align}
    {\boldsymbol{\hat x}_l'} &= \mathop{ \mathrm{ RMSNorm } } (\boldsymbol{\hat x}_l) \notag
    \\ {\boldsymbol{H}}^{\text{\text{pre}}}_l &= \mathop{ \mathrm{ sigmoid } }\left({\alpha^{\text{pre}}_l}{\boldsymbol{\hat x}_l'}\boldsymbol{W}^{\text{\text{pre}}}_l + \boldsymbol{b}_l^{\text{pre}}\right) \notag
    \\ {\boldsymbol{H}}^{\text{\text{post}}}_l &= 2\cdot \mathop{ \mathrm{ sigmoid } }\left({\alpha^{\text{post}}_l}{\boldsymbol{\hat x}_l'}\boldsymbol{W}^{\text{\text{post}}}_l + \boldsymbol{b}_l^{\text{post}}\right) \notag 
    \\ \boldsymbol{a}_{l} &=\mathop{ \mathrm{ softmax } }\left({\alpha^{\text{res}}_l}{\boldsymbol{\hat x}_l'}\boldsymbol{W}^{\text{\text{res}}}_l + \boldsymbol{b}_l^{\text{res}}\right) \label{eq: a} 
    \\ {\boldsymbol{H}}^{\text{\text{res}}}_l &= \sum_{k=1}^{n!}  a_{l,k}\boldsymbol{P}_k \label{eq: convex comb}
\end{align}
where $\boldsymbol{W}^{\text{\text{pre}}}_l,\boldsymbol{W}^{\text{\text{post}}}_l\in \mathbb{R}^{nC\times n}$ and $\boldsymbol{W}^{\text{\text{res}}}_l\in \mathbb{R}^{nC\times n!}$ are learnable weight matrices in the $l$-th layer. Here $\boldsymbol{b}^{\text{pre}}_l, \boldsymbol{b}^{\text{post}}_l\in \mathbb{R}^{1\times n}$ and $\boldsymbol{b}^{\text{res}}_l\in \mathbb{R}^{1\times n!}$ are learnable bias. The terms ${\alpha^{\text{pre}}_l},{\alpha^{\text{post}}_l}$ and ${\alpha^{\text{res}}_l}$ are learnable scalars. 
The $\mathop{ \mathrm{ RMSNorm } }(\cdot)$ refers to the RMSNorm~\cite{rmsnorm}.

In practice, we first compute a dynamic weight vector $\boldsymbol{a}_{l}=(a_{l,1},\ldots,a_{l,n!})\in\mathbb{R}^{n!}$ via a linear layer with softmax activations. Recall that $n$ denotes the number of residual streams, which is $n=4$ in HC and \mHC~\cite{Zhu2024HyperConnections,xie25mhc}, so $n!=24$ is a small constant. To produce ${\boldsymbol{H}}^{\text{\text{res}}}_l$, Equation~\ref{eq: convex comb} is implemented via a matrix multiplication between $\boldsymbol{a}_l^{\text{res}}$ and a constant 0/1 matrix in $\mathbb{R}^{n!\times n^2}$, which is reshaped from the concatenation of all permutation matrices. 

Like HC and \mHC~\cite{xie2024residual,Zhu2024HyperConnections}, the additional FLOPs introduced by the residual connection are typically negligible compared to those of the main transformation $f(\cdot;\mathcal{W}_l)$. For instance, in Transformer architectures~\cite{vaswani2017attention}, $f(\cdot;\mathcal{W}_l)$ corresponds to the attention and MLP operator, which dominates the compute. 
Our key advantage in the computation, instead, is engineering-oriented: the construction can be implemented entirely with standard operators, avoiding reliance on specialized kernels for repeated iterations, and is thus more generally portable across frameworks.



\begin{figure*}[t]
\begin{minipage}{0.24\linewidth}\centering
    \includegraphics[width=\linewidth]{figures/hres_small.pdf}
    \textsf{S} model, per-matrix
\end{minipage}
\begin{minipage}{0.24\linewidth}\centering
    \includegraphics[width=\linewidth]{figures/hres_large.pdf}
    \textsf{L} model, per-matrix
\end{minipage}
\begin{minipage}{0.24\linewidth}\centering
    \includegraphics[width=\linewidth]{figures/hres_prod_small.pdf}
    \textsf{S} model, prod
\end{minipage}
\begin{minipage}{0.24\linewidth}\centering
    \includegraphics[width=\linewidth]{figures/hres_prod_large.pdf}
    \textsf{L} model, prod
\end{minipage}
\caption{\textbf{Column sums of ${\boldsymbol{H}}^{\text{res}}$.} We compute column sums for token-level ${\boldsymbol{H}}^{\text{res}}$ matrices and summarize their distribution with standard boxplots (points indicate outliers). \textbf{per-matrix}: statistics for individual ${\boldsymbol{H}}^{\text{res}}$ matrices. \textbf{prod}: statistics for the layer-wise product of ${\boldsymbol{H}}^{\text{res}}$ across all layers.}\label{fig:h_res}
\end{figure*}


\begin{figure*}[th]
    \centering
\begin{minipage}{0.49\linewidth}\centering
    \includegraphics[width=\linewidth]{figures/nu_small.pdf}
    \textsf{S} model
\end{minipage}
\hfill
\begin{minipage}{0.49\linewidth}\centering  \includegraphics[width=\linewidth]{figures/nu_large.pdf}
\textsf{L} model
\end{minipage}
\caption{\textbf{Distribution of $\log(1/\nu)$.} Distribution of the relative range $\log (1/\nu)$ (defined in \cref{eq:nu}) for \mHC before applying SK. Large values (e.g., $\log(1/\nu)>30$) suggest that 20 SK iterations may not converge well to a doubly stochastic matrix. }\label{fig:nu}
\end{figure*}

\section{Experiments}
\label{sec: exp}
To evaluate the effectiveness of \mHC-lite, we implement \mHC-lite in language models by replacing the original residual connections, and assess its impact on both training efficiency and model performance across various scales and datasets. Specifically, we adopt the nanoGPT framework~\cite{karpathy2022nanogpt} and adopt three model scales: \textsf{S} (6 layers, $\sim$45M parameters), \textsf{M} (12 layers, $\sim$0.12B parameters), and \textsf{L} (24 layers, $\sim$0.36B parameters). For training data, we use \texttt{OpenWebText} and \texttt{FineWeb-Edu}. Following the implementation in \cite{xie25mhc}, throughout this paper $n$ is set to $4$. Due to computational constraints, we use a relatively small number of training iterations (10,000 steps, approximately 1.3B tokens in total). Further details of the hyperparameters are provided in \Cref{sec:exp-details}.

\paragraph{Initialization.} We initialize the parameters in the HC/\mHC/\mHC-lite blocks so that, at initialization, each block reduces to an ordinary residual connection. Concretely, in all variants, ${\boldsymbol{W}}_l^\text{pre}$, ${\boldsymbol{W}}_l^\text{post}$, and ${\boldsymbol{W}}_l^\text{res}$ are initialized to zero, while $\alpha_l^\text{pre}$, $\alpha_l^\text{post}$, and $\alpha_l^\text{res}$ are initialized to $0.01$. The bias vectors ${\boldsymbol{b}}_l^\text{pre}$ and ${\boldsymbol{b}}_l^\text{post}$ are set to $-1$ in all entries except for a single entry set to $1$. For \mHC, ${\boldsymbol{b}}_l^\text{res}$ is set to $-8$ for all entries except the diagonal, which is set to $0$, so that after exponentiation it closely approximates the identity matrix. For \mHC-lite, ${\boldsymbol{b}}_l^\text{res}$ is set to $-8$ for all entries except the entry corresponding to the identity matrix, which is set to $0$, so that after the $\mathop{\mathrm{softmax}}$ operation the weights concentrate on the identity matrix.


\subsection{Performance and Training Stability}To verify whether \mHC-lite achieves improvements in model loss comparable to those of \mHC, we compare the final training and validation losses of models with different residual connection components in \Cref{tab:loss}. The results clearly demonstrate that \mHC-lite achieves performance on par with \mHC or even slightly better across all datasets and model scales. 

Furthermore, \Cref{fig:grad-norm} presents the gradient norm curves for a specific configuration (the \textsf{L} model trained on \texttt{FineWeb-Edu}). The results indicate that \mHC-lite exhibits the same stabilizing effect on training as \mHC. Moreover, a closer examination of the curves (\Cref{fig:grad-norm} right) reveals that the gradient norm of \mHC-lite is slightly lower than that of \mHC, further confirming its effectiveness in stabilizing training dynamics.


\subsection{Efficiency}
\begin{figure}[h]
    \centering
    \includegraphics[width=0.8\linewidth]{figures/token_per_sec.pdf}
\caption{\textbf{Token throughput during training.} We report training throughput in tokens/s, computed as the number of tokens per batch divided by the wall-clock time of each optimizer update and averaged over the entire training run. All experiments are run on a single node with 8$\times$ NVIDIA A100 80GB (SXM4) GPUs. Notice that the \mHC result is based on our PyTorch re-implementation and may underestimate the throughput of the specialized-kernel implementation in \citet{xie25mhc}, which is reported to incur only a $6.7\%$ overhead relative to HC.}
    \label{fig:token_per_second}
\end{figure}



We compare the computational efficiency of \mHC-lite to HC by measuring the average training throughput (number of tokens per second) on the \texttt{OpenWebText} dataset using the \textsf{M} model. Results are reported in \Cref{fig:token_per_second}. Unless otherwise noted, all methods are implemented by us in PyTorch under the same training setup.

We have also included the \mHC results in \Cref{fig:token_per_second}. It is important to note that \citet{xie25mhc} accelerates \mHC using a specialized kernel, which is not publicly available at the time of writing. Therefore, the \mHC throughput reported in \Cref{fig:token_per_second} is based on our PyTorch re-implementation and may underestimate the performance achievable with custom kernels. 

Even with this caveat, the authors of \mHC claimed that with their optimized \mHC implementation, \mHC still incurs a $6.7\%$ overhead relative to HC \cite{xie25mhc}, whereas \mHC-lite achieves higher throughput than HC even \emph{without any system-level optimization}. This result suggests that \mHC-lite is highly implementation-friendly, making it easy to integrate into existing training code and practical systems.




\subsection{Stability Analysis}

In this section, we address the following question: \emph{Are the ${\boldsymbol{H}}^{\text{res}}_l$ matrices in \mHC really as stable as claimed in \citet{xie25mhc}?} To answer this, we follow the methodology in Section~5.4 of \citet{xie25mhc} and assess how close ${\boldsymbol{H}}^{\text{res}}_l$ is to being doubly stochastic. However, rather than analyzing \emph{token-averaged} matrices as in \citet{xie25mhc}, we collect matrices at each token and compute statistics over the resulting population. We argue that this procedure more faithfully reflects the behavior of ${\boldsymbol{H}}^{\text{res}}_l$, since averaging across tokens can hide potential instability. Concretely, for the experiments in this section, we first take the trained model and then run it on the first 64 sequences of the training set (each of length 1024). At every layer and every token, we record ${\boldsymbol{H}}^{\text{res}}_l$ and other related matrices, and report statistics over all collected matrices.

We begin with the relative range $1/\nu$ (defined in \cref{eq:nu}). The theoretical analysis for the SK algorithm suggests that convergence can be poor when $\log(1/\nu)$ is significantly larger than the number of SK iterations. In \Cref{fig:nu}, we report the distribution of $1/\nu$ for \mHC before applying SK. The left and right panels of \Cref{fig:nu} present the results for a 6-layer model and 24-layer model respectively. 
The results show that the fixed number of iteration, 20 times, taken by \mHC, is indeed a reasonable choice for balancing the converge rate and running time. 
On the other hand, however, there are also a non-negligible fraction of outliers with $\log(1/\nu) > 30$, i.e., $1/\nu >10^{13}$, a regime in which 20 SK iterations may not converge well to the Birkhoff polytope. 
By comparing the left and right panels, we further find that the relative range $1/\nu$ is generally larger for deeper models. This implies that the fixed 20 SK iterations might not be generically sufficient for deeper networks. 

To show this issue more explicitly, we further directly examine the distribution of column sums of ${\boldsymbol{H}}^{\text{res}}_l$ for \mHC (\mHC-lite guarantees that ${\boldsymbol{H}}^{\text{res}}_l$ is strictly doubly stochastic).  
As shown in \Cref{fig:h_res}, although the median column sum for an individual ${\boldsymbol{H}}^{\text{res}}_l$ is typically close to $1$, there exist many outliers that deviate substantially from $1$. Moreover, when we consider the composition $\prod_l{\boldsymbol{H}}^{\text{res}}_l$ across layers, even the median can drift far from $1$. Similarly, by comparing the composition $\prod_l{\boldsymbol{H}}^{\text{res}}_l$ for 6-layer models and 24-layer models, we find that the deviation is more severe when a model scales up, which implies the potential risks of instability when a model further scales up.

In contrast, \mHC-lite does not rely on iterative normalization and therefore avoids convergence-related failure. For \mHC-lite, the perfect doubly stochasticity of ${\boldsymbol{H}}^{\text{res}}_l$ and its composition $\prod_l{\boldsymbol{H}}^{\text{res}}_l$ is guaranteed by construction via the Birkhoff-von Neumann theorem.





\section{Conclusion and Discussion}
\label{sec:conclusion}

In this work, we revisit \mHC's design of residual connections from the perspective of stability and system portability. The iterative SK algorithm requires specialized kernels for efficient execution, creating engineering barrier for generic adoption. Moreover, through both theoretical analysis and empirical evaluation, we find that due to \mHC's reliance on a finite steps of SK iterations, its residual matrices may significantly deviate from doubly stochasticity, when the SK algorithm fails to converge, introducing potential risks of stability.  
To address these limitations, we propose \textbf{\mHC-lite}, a simple, strong, and efficient alternative to \mHC, achieved by re-parameterizing doubly stochastic matrices based on the Birkhoff–von Neumann theorem. The re-parameterization enables us to skip the SK iterations entirely, removing the approximation gap and supporting the computation with only basic operators, making our method a drop-in replacement for classical residual architectures, offering guaranteed robustness without sacrificing ease of deployment.

The design of \mHC-lite verifies a simple but powerful principle: exactness, when attainable, is often the most efficient form of approximation.
This shift from ``projection'' to ``reparameterization'' ensures the constraint hold by construction, eliminating approximation gaps (such as those induced by finitely many Sinkhorn--Knopp iterations) while enabling potentially more efficient implementations. 

\paragraph{On The Computational Efficiency of \mHC-lite for Larger $n$.}
An astute reader might notice that, although \mHC performs well when $n = 4$, its space and time complexity grow exponentially with $n$, raising potential concerns about the efficiency of this method when $n$ is larger. Here, we make two observations: 1) in the original HC paper~\cite{Zhu2024HyperConnections}, the authors conducted extensive ablation studies demonstrating that $n=4$ is indeed an superior choice in practice; 2) even if a larger $n$ is required, we can readily reduce the computational cost by sampling a subset of permutation matrices rather than including all of them. This is equivalent to restricting the feasible region to a subset of the Birkhoff polytope. The resulting residual matrix remains guaranteed to be doubly stochastic, while the computational budget can be tuned by controlling the number of sampled permutations.




\newpage 
% \section*{Accessibility}

% Authors are kindly asked to make their submissions as accessible as possible
% for everyone including people with disabilities and sensory or neurological
% differences. Tips of how to achieve this and what to pay attention to will be
% provided on the conference website \url{http://icml.cc/}.

% \section*{Software and Data}

% If a paper is accepted, we strongly encourage the publication of software and
% data with the camera-ready version of the paper whenever appropriate. This can
% be done by including a URL in the camera-ready copy. However, \textbf{do not}
% include URLs that reveal your institution or identity in your submission for
% review. Instead, provide an anonymous URL or upload the material as
% ``Supplementary Material'' into the OpenReview reviewing system. Note that
% reviewers are not required to look at this material when writing their review.

% Acknowledgements should only appear in the accepted version.
% \section*{Acknowledgements}

% \textbf{Do not} include acknowledgements in the initial version of the paper
% submitted for blind review.

% If a paper is accepted, the final camera-ready version can (and usually should)
% include acknowledgements.  Such acknowledgements should be placed at the end of
% the section, in an unnumbered section that does not count towards the paper
% page limit. Typically, this will include thanks to reviewers who gave useful
% comments, to colleagues who contributed to the ideas, and to funding agencies
% and corporate sponsors that provided financial support.

% \section*{Impact Statement}

% Authors are \textbf{required} to include a statement of the potential broader
% impact of their work, including its ethical aspects and future societal
% consequences. This statement should be in an unnumbered section at the end of
% the paper (co-located with Acknowledgements -- the two may appear in either
% order, but both must be before References), and does not count toward the paper
% page limit. In many cases, where the ethical impacts and expected societal
% implications are those that are well established when advancing the field of
% Machine Learning, substantial discussion is not required, and a simple
% statement such as the following will suffice:

% ``This paper presents work whose goal is to advance the field of Machine
% Learning. There are many potential societal consequences of our work, none
% which we feel must be specifically highlighted here.''

% The above statement can be used verbatim in such cases, but we encourage
% authors to think about whether there is content which does warrant further
% discussion, as this statement will be apparent if the paper is later flagged
% for ethics review.

% In the unusual situation where you want a paper to appear in the
% references without citing it in the main text, use \nocite
\nocite{langley00}

\bibliography{example_paper}
\bibliographystyle{icml2026}

%%%%%%%%%%%%%%%%%%%%%%%%%%%%%%%%%%%%%%%%%%%%%%%%%%%%%%%%%%%%%%%%%%%%%%%%%%%%%%%
%%%%%%%%%%%%%%%%%%%%%%%%%%%%%%%%%%%%%%%%%%%%%%%%%%%%%%%%%%%%%%%%%%%%%%%%%%%%%%%
% APPENDIX
%%%%%%%%%%%%%%%%%%%%%%%%%%%%%%%%%%%%%%%%%%%%%%%%%%%%%%%%%%%%%%%%%%%%%%%%%%%%%%%
%%%%%%%%%%%%%%%%%%%%%%%%%%%%%%%%%%%%%%%%%%%%%%%%%%%%%%%%%%%%%%%%%%%%%%%%%%%%%%%
\newpage
\appendix
\onecolumn

\appendix
\onecolumn

\section{What’s the Real Difference between Conventional PEs, NoPE, and RePo?\label{app:comp}}

In the background section (\S\ref{sec:bg}), we use \textsc{RoPE} as a representative example to illustrate how conventional positional encoding methods rely on a strict linear pattern to assign positional information to tokens in the context.


Recently, researchers have found that the causal mask in the attention mechanism enables LLMs to implicitly learn positional information, and that removing explicit positional encoding can even achieve superior performance on structured data and long-context tasks. This approach is referred to as the NoPE method \cite{kazemnejad2023impact, yang2025rope, wang-etal-2024-length, barbero2025round}. We argue that the attention score of NoPE can be reformulated within the RoPE framework by assigning a constant positional value $a$:
\begin{align}
\mathbf{A}_{i,j}^{\text{NoPE}} &= \boldsymbol{q}_i^\top \boldsymbol{k}_j \nonumber \\
&= \boldsymbol{q}i^\top g\theta(0)\boldsymbol{k}_j \nonumber \\
&= \boldsymbol{q}i^\top g\theta(a - a)\boldsymbol{k}_j,
\end{align}
where $a$ denotes a uniform position value for all tokens, yielding a constant rotation matrix $g_\theta(0)$. Thus, under this reformulation, the key difference between RoPE and NoPE lies solely in how positions are assigned.

\begin{wraptable}{R}{0.4\columnwidth}
    \centering
    % \tiny
    \caption{Comparison between different methods. In RoPE-like methods, $g_\theta$ generates a rotation matrix based on a distance. The $j-i$ is the distance between $x_j$ and $x_i$, $g_\theta(0)$ is a constant rotation, and $z_j$ and $z_i$ are predicted positions by $f_\phi$ (Eq. \ref{eq:repo}).}
    \label{tab:method_comp}
        \begin{tabular}{lc}\toprule
        Method & Attention Score \\\midrule
        Linear (e.g., \textsc{RoPE}) & $\boldsymbol{q}_i^\top g_\theta(j-i)\boldsymbol{k}_j$ \\[0.5em]
        Constant (e.g.,\textsc{NoPE}) & $\boldsymbol{q}_i^\top g_\theta(0)\boldsymbol{k}_j$ \\[0.5em]
        \implname & $\boldsymbol{q}_i^\top g_\theta (z_j - z_i) \boldsymbol{k}_j$ \\\bottomrule
        \end{tabular}
\end{wraptable}


In addition, LLMs with interpolated  NoPE and RoPE layers \cite{yang2025rope, meta2025llama, barbero2025round} have been widely used architectures, which can be explained as hybrid position assignment strategies. In contrast to previous works that empirically configure the hybrid system with hyper-parameters, our
\implname shows higher expressiveness, as it can dynamically determine whether to adopt the conventional linear, NoPE-like constant, or hybrid position assignment for tokens in a given context. A comparison among the three approaches is summarized in Table~\ref{tab:method_comp}.
As explained in \S\ref{sec:bg},  when $z_j = z_i$, \implname effectively reduces to the NoPE pattern with the constant $z_j = z_i = a$. In contrast, if $z_j > z_i$ for $j > i$, it indicates that \implname adopts positional relationship similar to the conventional linear style, e.g., the strategt used in RoPE.
In our experiments and analyses, we will demonstrate that an LLM may dynamically select between constant and linear position assignments, or hybridize them with \implname module. Notably, although we use RoPE for the comparison, linear position assignment is widely adopted in conventional positional encoding methods \cite{vaswani2017attention, gehring2017convolutional, press2021train, li2025seqpe}, and our findings can be readily extended to these approaches.



\section{Details of Experiments\label{app:exp}}

\subsection{Extrapolation}
We use the following hyper-parameters to extend the context:
\begin{enumerate}
    \item 8K Tokens: \texttt{\{"rope\_type": "yarn", "factor": 2.0, "original\_max\_position\_embeddings": 4096\}}
    \item 16K Tokens: \texttt{\{"rope\_type": "yarn", "factor": 4.0, "original\_max\_position\_embeddings": 4096\}}
\end{enumerate}
We use the setting for ``16K Tokens'' for all the experiments on LongBench (Table \ref{tab:lc_longbench}).

\subsection{General Tasks}
We use the following task ids in \texttt{olmes} for the evaluation in Table \ref{tab:general}: \texttt{arc\_challenge:rc::large}, \texttt{arc\_easy:rc::olmes}, \texttt{boolq:rc::large}, \texttt{coqa::large}, \texttt{drop::large}, \texttt{hellaswag:rc::large}, \texttt{mmlu\_pro:cot::none}, \texttt{triviaqa::olmes}




\begin{wrapfigure}{R}{0.4\columnwidth}
    \centering
    % \includegraphics[width=\linewidth]{figs/ablation.pdf}
    \includegraphics[width=\linewidth]{figs/ablation.png}
    \caption{Sensitivity to the starting layer of \implname (i.e. $l=3,5,7$). We validate on the NIAH subtask of RULER benchmark and MMLUPro of general benchmarks.}
    \label{fig:ablation}
\end{wrapfigure}




\subsection{Ablation Study \label{app:ablation}}

As shown in Figure \ref{fig:ablation}, we evaluate the sensitivity of model performance to the starting layer of \implname, where $l=5$ indicates that \implname is applied beginning from the 5th layer of the LLM, while the vanilla \textsc{RoPE} is used for the lower layers. We conduct experiments on two subtasks, NIAH and MMLUPro. The results show that overall performance is robust to this hyperparameter. However, increasing $l$ slightly improves performance on general benchmarks while negatively affecting performance on NIAH. The results are consistent on other evaluation benchmarks.



\section{Preliminary Experiments \label{app:pre}}




\begin{figure*}[]
    \begin{center}
        \includegraphics[width=0.95\textwidth]{figs/layer_visual.png}
    \end{center}
    \caption{Visualization of predicted positions from a 4-layer GPT-2 model in the reversal task. The \textcolor{Cerulean}{area with blue background color} indicates input context, while the \textcolor{Apricot}{orange region} is the generated sequence. We use A-K to replace the real tokens to save space for illustration. The x and y-axis represent the input order and predicted position $z$ of a token, respectively.
    \label{fig:visual}}
\vspace{-1em}
\end{figure*}

In this experiment, we train a small-scale language model on a purposefully selected synthetic task, namely text reversal, to determine whether \implname can re-position tokens in the context.


In the text reversal task, a model is prompted to generate a given sequence of tokens $\boldsymbol{x} = \{x_1, x_2, \dots, x_L\}$ in a reversed order $\boldsymbol{x}^\prime = \{x_L, x_{L-1}, \dots, x_1\}$. \textit{Locality bias} does not apply here because the distance between a generated token and its corresponding dependent input token grows linearly as the generation proceeds. It is interesting to investigate whether the \implname method can learn beneficial re-positioning patterns from the task.

\subsection{Setup} We use the data and train/dev/test splits provided in \citet{kazemnejad2023impact} for the text reversal task. The sequence lengths $L$ in training are between $[2, 20]$, while we use the sequence lengths between $[2, 30]$ for evaluation.  The input sequence $\boldsymbol{x}$ is constructed from a fixed set of subwords that are shared across the three datasets, without regard to grammatical or semantic structure \cite{kazemnejad2023impact}.


We train a GPT-2 model with 4 layers for this task. We use NoPE, RoPE, and \implname\footnote{Notably, \implname functions solely as a position prediction module. For brevity, however, mentioning \implname implies the use of RoPE encoding together in this work.} methods to train the model. For the \implname method, we shared the parameters of $f_\phi$ for all the attention heads in each layer. All training hyper-parameters are set as in \citet{kazemnejad2023impact}. 


\begin{wrapfigure}{R}{0.5\columnwidth}
    \centering
    \includegraphics[width=\linewidth]{figs/synthetic_task.png}
    \caption{Performance on the text reversal task. We report the accuracy on all lengths of input sequences.}
    \label{fig:synthetic_task}
\end{wrapfigure}

\subsection{Findings}
As shown in Figure \ref{fig:synthetic_task}, due to the simplicity of the text reversal task, all the models achieve nearly perfect performance on short in-domain sequences ($L \leq 20$). However, when testing on examples with longer-range dependencies, i.e. $L > 20$, our \implname method demonstrates superior performance compared to both NoPE and RoPE. 

We further investigate the re-positioning patterns  learned by \implname that contribute to this performance gain. As illustrated in Figure \ref{fig:visual}, we visualize the predicted positions across different layers of the trained model and observe several intriguing patterns. The overall distribution of predicted positions is remarkably distinct from the pre-defined positional indices (e.g., 1 to 27). Specifically, we observe a mirror effect in layers 0, 2, and 3, where pairs of reversal tokens are assigned the same position indices. Additionally, we identify hybrid positional patterns across different parts of the context. For example, from layers 1 to 3, the model adopts a NoPE-like pattern for the opening tokens, assigining nearly identical position indices to tokens in the phrase “Reverse the following words:”, while exhibiting patterns with mirror symmetry for the input and output sequences. Notably, we did not introduce any inductive bias for this task; all patterns emerged in a purely data-driven manner. These intriguing patterns motivate us to further investigate \implname on more general datasets. 




\section{Case Study\label{app:case_study}}
As shown in Figure \ref{fig:case_study}, we visualize the positions assigned by \implname when testing on the MMLUPro benchmark \cite{wang2024mmlu} with few-shot examples. We observe that \implname learns distinct patterns across different layers and attention heads. Interestingly, as shown in Figure \ref{fig:case2}, the patterns of assigned positions roughly align with the semantic segmentation of the few-shot examples, demonstrating that \implname is capable of capturing the structure of the input context. Additionally, we find that some positions assigned by \implname are negative values, as shown in Figure \ref{fig:case3}. Those negative positions can be interpreted as rotations in a reversed direction under the RoPE framework. There also exist some outlier positions in the figures. Upon inspection, we find that they correspond to non-informative punctuation marks and function words, such as ``.'' and ``such.''

\begin{figure}[htbp]
    \centering
    % --- First Subfigure ---
    \begin{subfigure}{\linewidth}
        \centering
        \includegraphics[width=0.9\linewidth]{figs/l5h1.png}
        \caption{Layer 5 \& Attention Head 1}
        \label{fig:case1}
    \end{subfigure}
    
    \vspace{1em} % Add vertical space between the images
    
    % --- Second Subfigure ---
    \begin{subfigure}{\linewidth}
    \centering
        \includegraphics[width=0.9\linewidth]{figs/l8h0.png}
        \caption{Layer 8 \& Attention Head 0}
        \label{fig:case2}
    \end{subfigure}


    \begin{subfigure}{\linewidth}
    \centering
        \includegraphics[width=0.9\linewidth]{figs/l13h3.png}
        \caption{Layer 13 \& Attention Head 3}
        \label{fig:case3}
    \end{subfigure}


    \caption{Visualization of positions assigned by \implname. The \implname is continuously trained on general data. The visualization data is from MMLUPro with few-shot examples. Symbols in \textcolor{orange}{orange} belong to the prompt, while symbols in \textcolor{blue}{blue} and \textcolor{red}{red} represent questions and answers in few-shot examples.}
    \label{fig:case_study}
\end{figure}


\end{document}

